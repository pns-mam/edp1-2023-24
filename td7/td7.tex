\documentclass[12pt,a4paper]{article}

%\usepackage[T1]{fontenc} % Pour la bonne cesure du francais
\usepackage{amsmath} % Pour les symboles complementaire comme les matrices !
\usepackage{amssymb}
\usepackage{verbatim}
\usepackage{epsfig}
%\usepackage{/home/cohen/fortran/graphics/GGGraphics/GGGraphics}
%\usepackage{D:/GGGraphics/GGGraphics}

\newtheorem{theorem}{Theorem}
\newtheorem{corollary}[theorem]{Corollary}
\newtheorem{example}{Example}
\newtheorem{rem}{\noindent\textbf{\textit {Remarque\,}}}
\newcommand{\qed}{\hfill$\qedsquare$\goodbreak\bigskip}

\def\e{{\mathchoice{\hbox{\mathbb{R}m e}}{\hbox{\mathbb{R}m e}}%
        {\hbox{\mathbb{R}m \scriptsize e}}{\hbox{\mathbb{R}m \tiny e}}}}
        
\advance\voffset by -35mm \advance\hoffset by -25mm
\setlength{\textwidth}{175mm} \setlength{\textheight}{260mm}
\pagestyle{empty}

\begin{document}

\noindent {\large Universit\'e C\^ote d'Azur} \hfill Polytech Nice Sophia (PNS)\\
\noindent Math\'ematiques Appliqu\'ees et Mod\'elisation (MAM4) \hfill 2023-24 \\

\hrule

\bigskip
\bigskip

\begin{center}{\bf \'Equations aux d\'eriv\'ees partielles --
TD 7}\end{center}

\bigskip

\noindent On rappellera la d\'efinition des principaux op\'erateurs
diff\'erentiels (ici $u$ est un champ scalaire et $\mathbf{w} =
(w_x,w_y,w_z)$ un champ vectoriel). Par convention, on notera en caract\`ere gras les champs vectoriels et en
caract\`ere simple les champs scalaires. On supposera toutes les fonctions suffisamment r\'eguli\`eres afin de
  pouvoir effectuer les op\'erations d'int\'egration indiqu\'ees, au sens classique.
$$
\begin{array}{lcl}
\nabla u &=& \displaystyle\left(\frac{\partial u}{\partial
    x},\frac{\partial u}{\partial y }, \frac{\partial u}{\partial
    z}\right)^T \, (\mbox{Gradient}), \\[2ex]
\Delta u &=& \displaystyle \frac{\partial^2 u}{\partial
    x^2}+\frac{\partial^2 u}{\partial y^2 }+\frac{\partial^2 u}{\partial
    z^2} \, (\mbox{Laplacien}), \\[2ex]
\nabla\cdot \mathbf{w} &=&  \displaystyle\frac{\partial w_x}{\partial x} +
\frac{\partial w_y}{\partial y} +\frac{\partial w_z}{\partial z}\,  (\mbox{Divergence}),\\[2ex]
 \nabla\times \mathbf{w} &=&  \displaystyle\left(\frac{\partial w_z}{\partial y}
   -\frac{\partial w_y}{\partial z},\frac{\partial w_x}{\partial z}
   -\frac{\partial w_z}{\partial x}, \frac{\partial w_y}{\partial x}
   -\frac{\partial w_x}{\partial y}  \right)^T \, (\mbox{Rotationnel}).
\end{array}
$$
\'Etant donn\'e un domaine (volume) $\Omega$, on admettra par la
suite la formule de Green
$$
\int_{\Omega} \frac{\partial u}{\partial x_i} d\mathbf{x}
= \int_{\partial\Omega}u n_i \, d\sigma.
$$
o\`u $\partial\Omega$ repr\'esente la fronti\`ere de $\Omega$ et
$\mathbf{n}$ est la normale ext\'erieure à cette fronti\`ere, $n_i$
la $i$-\`eme composante de cette normale et $x_i$ la $i$-\`eme
coordonn\'ee de $\mathbf{x}$.\\

\noindent On cherchera \`a prouver une s\'erie de relations utiles
\begin{enumerate}
\item Pour les champs $u$ et $\mathbf{v}$ v\'erifier formellement les relations
$$
\begin{array}{c}
\displaystyle \nabla\cdot (\nabla u )\, = \, \Delta u,\quad \nabla\cdot (u\mathbf{v}) = u (\nabla\cdot \mathbf{v})+ \mathbf{v}
\cdot\nabla u,\quad \nabla\times (u\mathbf{v}) = u(\nabla\times \mathbf{v})+ \nabla u
 \times \mathbf{v}.
\end{array}
$$
En d\'eduire que pour deux fonctions $u$ et ${\phi}$ on a :
$$
\nabla\cdot\, (u\nabla \phi) \, = \, u \, \Delta \phi \, + \, \nabla u \cdot \nabla \phi
$$
\item Montrer que si $u$ et $v$ sont des champs scalaires d\'efinis
  sur $\displaystyle \mathbb{R}^n$, alors on a la formule suivante
  d'integration par parties :
$$
\int_\Omega \Delta u  \, v \, d\mathbf{x}= \, - \, \int_\Omega \nabla
u \, \cdot \, \nabla v \,d\mathbf{x}+ \, \int_{\partial \Omega}
\frac{\partial u}{\partial \mathbf{n}}  \, v \, d\sigma.
$$
o\`u la {\sl d\'eriv\'ee normale} est d\'efinie par $\displaystyle
\frac{\partial u}{\partial \mathbf{n}}= \nabla u \cdot \mathbf{n}$. On
montrera d'abord que
$$
\int_\Omega \frac{\partial u}{\partial x_i}  \, v \, d\mathbf{x}= \, - \, \int_\Omega 
u \frac{\partial v}{\partial x_i}\,d\mathbf{x}+ \, \int_{\partial \Omega}
u v n_i \, d\sigma.
$$
qui est une consequence directe de la formule de Green.
\item A partir de la formule de Green, montrer la formule de Stokes:
$$
\int_{\Omega} (\nabla\cdot \mathbf{w} )\phi\, d\mathbf{x} = -\int_{\Omega}
\mathbf{w}\cdot \nabla \phi \,d\mathbf{x} + \int_{\partial\Omega} (\mathbf{w}\cdot
\mathbf{n}) \phi \,d\sigma.
$$ 
et en d\'eduire la formule de la divergence ou la formule de Gauss:
$$
\int_{\Omega}\nabla\cdot \mathbf{w}\, d\mathbf{x} =
\int_{\partial\Omega} \mathbf{w} \cdot \mathbf{n}\, d\sigma.
$$
\item A partir de la formule de Green, montrer la formule
$$
\int_{\Omega} (\nabla\times \mathbf{w})\, \phi d\mathbf{x} = \int_{\Omega}
\mathbf{w}\times \nabla \phi \,d\mathbf{x} - \int_{\partial\Omega} (\mathbf{w}\times
\mathbf{n})\phi\, d\sigma.
$$ 
et en d\'eduire la formule du rotationnel:
$$
\int_{\Omega}\nabla\times \mathbf{w}\, d\mathbf{x} =-
\int_{\partial\Omega} \mathbf{w} \times \mathbf{n}\, d\sigma.
$$
\item Si $f$ est une fonction continue sur $[0; 1]$, montrer que le
  probl\`eme de Laplace en une dimension d'espace a une
solution unique dans $C^2([0; 1])$  donn\'ee par la formule
$$
u(x) = x\int_0^1f(s)(1-s)ds - \int_0^x f(s)(x-s)ds, x\in[0,1].
$$
\item 
Montrer que le Laplacien en deux dimensions d'espace peut s'\'ecrire
en coordonn\'ees polaires de la fa\c{c}on suivante:
$$
\Delta u =\frac{\partial^2 u}{\partial r^2}
        +\frac{1}{r}\frac{\partial u}{\partial
          r}+\frac{1}{r^2}\frac{\partial^2 u}{\partial \theta^2}
$$
On rappelle que la relation entre les coordonn\'ees polaires $(r,\theta)$ et
cartesiennes $(x,y)$ est
$$
x=r\cos(\theta),\, y=r\sin(\theta).
$$
\end{enumerate}

%\bigskip
%\hrule
%\noindent{\bf Evaluation du cours \'Equations aux D\'eriv\'ees Partielles :}
%\begin{itemize}
%\item[$\bullet$] Un contr\^ole \'ecrit le vendredi $23$ Octobre (pendant la s\'eance de cours). 
%\item[$\bullet$] Une note de devoir maison/projet (présentation orale le 4 décembre).
%\item[$\bullet$] Un examen \'ecrit pendant la session d'examen. 
%\end{itemize}
%La note finale est : $30\%$(note contr\^ole) $+$
%$30\%$(note devoir/projet) $+$
%$40\%$(note examen).
\end{document}
