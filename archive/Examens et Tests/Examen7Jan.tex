\documentclass[12pt,a4paper]{article}

%\usepackage[T1]{fontenc} % Pour la bonne cesure du francais
\usepackage{amsmath} % Pour les symboles complementaire comme les matrices !
\usepackage{amssymb}
\usepackage{verbatim}
\usepackage{epsfig}
%\usepackage{/home/cohen/fortran/graphics/GGGraphics/GGGraphics}
%\usepackage{D:/GGGraphics/GGGraphics}

\newtheorem{theorem}{Theorem}
\newtheorem{corollary}[theorem]{Corollary}
\newtheorem{example}{Example}
\newtheorem{proposition}{Proposition}
\newtheorem{rem}{\noindent\textbf{\textit {Remarque\,}}}
\newcommand{\qed}{\hfill$\qedsquare$\goodbreak\bigskip}

\def\e{{\mathchoice{\hbox{\mathbb{R}m e}}{\hbox{\mathbb{R}m e}}%
        {\hbox{\mathbb{R}m \scriptsize e}}{\hbox{\mathbb{R}m \tiny e}}}}
        
\advance\voffset by -35mm \advance\hoffset by -25mm
\setlength{\textwidth}{175mm} \setlength{\textheight}{260mm}
\pagestyle{empty}

\begin{document}

\noindent {\large Universit\'e de Nice-Sophia Antipolis} \hfill EPU\\
\noindent MAM4 - \'Equations aux d\'eriv\'ees partielles \hfill 
7 Janvier 2012 \\

\hrule

\bigskip
\bigskip
%\parskip 6pt
\centerline {\large \sc Examen EDP. Dur\'ee : 2H00}
\vspace{1cm}

\hrule
\vspace{0.5cm}
\noindent {\sl Documents autoris\'es: les documents de cours et TD. Justifier vos reponses et commenter
  les programmes d'une fa\c{c}on concise
et claire. On pourra consid\'erer comme acquis les d\'eveloppements
d\'ej\`a faits ailleurs \`a condition de bien situer la source
(cours, no. s\'erie exercices, no. exercice)}\\

{\bf Probl\`eme 1}. Consid\'erons l'\'equation d'advection dans le domaine born\'e $(0,1)$:
\begin{equation}\label{eq:transp}
\begin{cases}
\displaystyle\frac{\partial u}{\partial t}+V\frac{\partial u}{\partial
  x}=0,\, \forall (x,t)\in(0,1)\times\mathbb{R}^+_*,
\end{cases}
\end{equation}
avec $u(x, 0) = u_0$, $u$ et $u_0$ p\'eriodiques de p\'eriode 1 et la
vitesse de convection positive $V>0$. On souhaite discr\'etiser \eqref{eq:transp}
en utilisant le sch\'ema explicite d\'ecentr\'e amont.
\begin{enumerate}
\item Pr\'eciser le sch\'ema de discr\'etisation et d\'eterminer l'\'equation \'equivalente associ\'ee,
  apr\'es l'\'evaluation du terme dominant de son erreur de
  troncature. Montrer que la discr\'etisation de cette \'equation
  \'equivalente donne un nouveau sch\'ema qui peut s'\'ecrire sous la
  forme:
\begin{equation}\label{eq:equiv}
\frac{u_j^{n+1}-u_{j}^{n}}{\Delta t}+V
\frac{u_{j}^{n}-u_{j-1}^{n}}{\Delta x}-\nu^*\frac{u_{j+1}^{n}-2u_j^n+u_{j-1}^{n}}{\Delta x^2}=0.
\end{equation}
avec un $\nu^*$ que l'on pr\'ecisera.
\item D\'eterminer les conditions de stabilit\'e $L^{\infty}$ et $L^2$
  pour le sch\'ema \eqref{eq:equiv}. Est-ce qu'elles sont
  diff\'erentes de celles pour le sch\'ema d\'ecentr\'e initial?
\end{enumerate}
{\bf Probl\`eme 2}. Consid\'erons l'\'equation d'advection-diffusion dans le domaine born\'e $(0,1)$:
\begin{equation}
\displaystyle\frac{\partial u}{\partial t}+V\frac{\partial u}{\partial
  x}-\nu\frac{\partial^2u}{\partial x^2}=0,\, \forall (x,t)\in(0,1)\times\mathbb{R}^+_*,
\end{equation}
avec $u(x, 0) = u_0$, $u$ et $u_0$ p\'eriodiques de p\'eriode 1.
Consid\'erons le sch\'ema centr\'e suivant:
$$
\frac{u_j^{n+1}-u_{j}^{n}}{\Delta t}+V
\frac{u_{j+1}^{n+1}-u_{j-1}^{n+1}}{2\Delta x}-\nu\frac{u_{j+1}^{n}-2u_j^n+u_{j-1}^{n}}{2\Delta x^2}-\nu\frac{u_{j+1}^{n+1}-2u_j^{n+1}+u_{j-1}^{n+1}}{2\Delta x^2}=0.
$$
\begin{enumerate}
\item Montrer que le sch\'ema est consistant et d\'eterminer son ordre.
\item \'Etudier la stabilit\'e $L^2$ du sch\'ema. Que se passe-t-il lorsque
$\nu \rightarrow 0$?
\end{enumerate}
{\bf Probl\`eme 3}. On cherche \`a r\'esoudre le probl\`eme aux
limites le suivant dans le carr\'e $\Omega=[-1,1]^2$, 
\begin{equation}\label{eq:tube}
\left\{\begin{array}{lcl}
-\Delta u &=& f,\,\text{dans }\Omega\\
\frac{\partial u}{\partial \mathbf{n}}(x,\pm 1)&=&1,\,\forall x\in ]-1,1[,\\
u(\pm 1,y)&=&0,\,\forall y\in  ]-1,1[.
\end{array}\right.
\end{equation}
\begin{enumerate}
\item D\'eterminer la formulation variationnelle (FV) associ\'ee dans un
  espace $X$ que l'on pr\'ecisera.
\item Montrer que $u$ est solution de (FV) ssi elle minimise sur $X$ une fonctionnelle $E(v)$ que l'on pr\'ecisera.
\item Peut-on \'etablir l'unicit\'e de la solution \`a partir de cette
  \'equivalence ? (on suppose qu'il existent deux solutions. etc...)
\end{enumerate}
 \texttt{Indication}: apr\`es la
  multiplication par la fonction test $v$ et integration par parties,
  on constatera que l'integrale sur la fronti\`ere $\partial\Omega$
  se d\'ecompose en $4$ parties correspondant aux cot\'es du carr\'e
  et o\`u les conditions aux limites sont diff\'erentes.\\

\noindent {\bf Probl\`eme 4}. On s'interesse au calcul d'une approximation de la solution $u:[0,1]\rightarrow\mathbb{R}$ du probl\`eme:
\begin{equation}\label{prob}
\left\{\begin{array}{l}
-u''(x) = f(x),\\
u(0)=u(1)=0.
\end{array}\right.
\end{equation}
Une discr\'etisation par diff\'erences
finies de l'\'equation, en utilisant une partition uniforme de
l'intervalle $[0,1]$, donn\'ee par les points $x_j=j/n,\,j=1,...,n-1$
conduit \`a un syst\`eme lin\'eaire $A \mathbf{u} = b$
o\`u $\mathbf{u}\in\mathbb{R}^{n-1}$ est une vecteur de composantes $u_j$
repr\'esentant la solution discr\`ete. Cette derni\`ere approche la
solution exacte aux $n-1$ points int\'erieurs $x_j,\,j=1,..,n-1$ (on
dit que $u_j$ est une approximation de $u(x_j)$ et
$u_0=u_{n}=0$, ceci traduisant le fait que $u(0)=u(1)=0$).  La matrice $A\in{\cal M}_{n-1}(\mathbb{R})$ et le second membre $b\in \mathbb{R}^{n-1}$ sont respectivement:
\begin{equation}\label{matrice}
A=n^2\left(\begin{array}{ccccc}
2 & -1 & 0 & \hdots & 0\\
-1 & 2 & -1 & \ddots & \vdots \\
0 & \ddots & \ddots & \ddots & 0 \\
\vdots & \ddots & -1 & 2 & -1 \\
0 & \hdots & 0 & -1 & 2    
\end{array}\right),\,b = \left(\begin{array}{c}f(x_1)\\ \vdots \\ f(x_{n-1})\end{array}\right).
\end{equation}
\begin{enumerate}
\item Initialisation.
\begin{itemize}
\item \'Ecrire une fonction \texttt{laplaceD(n)} construisant la matrice $A$, dont l'argument d'entr\'ee est $n$ et l'argument de sortie est $A$.
\item \'Ecrire une fonction ayant pour argument $n$ et $f$ et qui construit le vecteur $b$.
\end{itemize}
\item Validation.
\begin{itemize}
\item Donner la solution exacte $\tilde u^e(x)$ du probl\`eme
  (\ref{prob}), quand la fonction $f$ est constante \'egale \`a $1$. 
\item \'Ecrire une fonction construisant le vecteur $\mathbf{u}^e = (\tilde
  u^e(x_1),\hdots,\tilde u^e(x_{n-1}))^t$  qui donne la solution
  exacte aux points $x_k$. 
\item R\'esoudre le syst\`eme $A \mathbf{u} =
  b$. Comparer $\mathbf{u}$ et $\mathbf{u}^e$ en les dessinant sur le
  m\^eme graphique. Expliquer quel genre de r\'esultat on attend.
\end{itemize}
\item Convergence de la m\'ethode. On prend comme solution exacte et
  second membre les fonctions suivantes
$$
\tilde u^e(x)= (x-1)\sin(10x),\,f(x) = -20\cos(10x)+100(x-1)\sin(10x).
$$
\begin{itemize}
\item Construire d'abord deux fonctions correspondant \`a $\tilde u^e(x)$
et $f(x)$. 
\item Repr\'esenter la norme de l'erreur entre la solution
approch\'ee (obtenue par la m\'ethode des diff\'erences finies) et
celle exacte $\mathbf{u} - \mathbf{u}^e$ en fonction de $n$
en \'echelle logarithmique. Expliquer quel genre de r\'esultat on
attend. 
\end{itemize}
\texttt{Indication}: on pourra prendre par exemple la norme
$L^{\infty}$ ($\|u\|_{L^{\infty}} = \max_i |u_i| $) calcul\'ee par
exemple en \texttt{Scilab} par \texttt{norm(x,'inf')}
\end{enumerate}
\end{document}

