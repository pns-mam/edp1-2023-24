\documentclass[12pt,a4paper]{article}

%\usepackage[T1]{fontenc} % Pour la bonne cesure du francais
\usepackage{amsmath} % Pour les symboles complementaire comme les matrices !
\usepackage{amssymb}
\usepackage{verbatim}
\usepackage{epsfig}
%\usepackage{/home/cohen/fortran/graphics/GGGraphics/GGGraphics}
%\usepackage{D:/GGGraphics/GGGraphics}

\newtheorem{theorem}{Theorem}
\newtheorem{corollary}[theorem]{Corollary}
\newtheorem{example}{Example}
\newtheorem{proposition}{Proposition}
\newtheorem{rem}{\noindent\textbf{\textit {Remarque\,}}}
\newcommand{\qed}{\hfill$\qedsquare$\goodbreak\bigskip}

\def\e{{\mathchoice{\hbox{\mathbb{R}m e}}{\hbox{\mathbb{R}m e}}%
        {\hbox{\mathbb{R}m \scriptsize e}}{\hbox{\mathbb{R}m \tiny e}}}}
        
\advance\voffset by -35mm \advance\hoffset by -25mm
\setlength{\textwidth}{175mm} \setlength{\textheight}{260mm}
\pagestyle{empty}

\begin{document}

\noindent {\large Universit\'e de Nice-Sophia Antipolis} \hfill EPU\\
\noindent MAM4/SI4 - \'Equations aux D\'eriv\'ees Partielles \hfill 
28 Novembre, 2012 \\

\hrule

\medskip
%\bigskip
%\parskip 6pt
\centerline {\large \sc Test \'Equations aux D\'eriv\'ees Partielles. Dur\'ee : 1H30}
\vspace{0.5cm}

\noindent {\bf NOM \& Pr\'enom} :
........................................................ {\bf Groupe TD} : 
\vspace{1cm}

\hrule
\vspace{0.5cm}
\noindent {\sl Documents autoris\'es: uniquement les documents cours et TD distribu\'es. R\'eponses \`a r\'ediger sur la feuille d'\'enonc\'e (il n'en sera distribu\'e qu'une), apr\`es avoir fait vos
  exercices/essais au brouillon. Justifier vos reponses et commenter
  les programmes Scilab d'une fa\c{c}on concise
et claire. \\On pourra consid\'erer comme acquis les d\'eveloppements
d\'ej\`a faits ailleurs \`a condition de bien situer la source
(cours, no. s\'erie exercices, no. exercice)}\\

\noindent En absence de pr\'ecisions suppl\'ementaires, on discr\'etise
toujours le domaine en utilisant un maillage r\'egulier $(x_j,t_n)=(j\Delta x,n\Delta t)$,  $\forall n\ge 0,\,j\in\{0,1,...,N+1\}$, $\Delta x=1/(N+1)$ et $\Delta t>0$.\\
\noindent {\bf Probl\`eme 1}. On consid\`ere l'\'equation de la chaleur :
\begin{equation} \label{eqn:chaleur}
\left\{
\begin{array}{rcl}
\displaystyle \frac{\partial u}{\partial t} \, - \nu\, \frac{\partial^2 u}{\partial x^2}  & =  & 0  \qquad \forall \, (x,t) \, \in \, \mathbb{R} \times \mathbb{R}^+ \\
\displaystyle u(x,0)  & =  & u^0(x)  \qquad \forall \, x \, \in \, \mathbb{R}
\end{array}
\right.
\end{equation}
et on se propose de la r\'esoudre en utilisant le sch\'ema num\'erique explicite suivant
\begin{equation} \label{eqn:schema1}
\displaystyle  \frac{u_j^{n+1} \, - \, u_j^{n-1}}{2 \Delta t} \, - \nu\,  \frac{u_{j+1}^{n} \,- \,  2 \, u_j^{n} \, + \, u_{j-1}^{n}}{\Delta x ^2 } \,  =  \,   0 \qquad (j,n) \, \in \, \mathbb{Z} \times \mathbb{N}^* .
\end{equation}
ainsi que sa version implicite
\begin{equation} \label{eqn:schema2}
\displaystyle  \frac{u_j^{n+1} \, - \, u_j^{n-1}}{2 \Delta t} \, - \nu\,  \frac{u_{j+1}^{n+1} \,- \,  2 \, u_j^{n+1} \, + \, u_{j-1}^{n+1}}{\Delta x ^2 } \,  =  \,   0 \qquad (j,n) \, \in \, \mathbb{Z} \times \mathbb{N}^* .
\end{equation}

\begin{enumerate}
\item Calculer l'ordre des sch\'emas (\ref{eqn:schema1})  et  (\ref{eqn:schema2}). ({\bf 3
    POINTS})
\newpage
\item Montrer que le sch\'ema explicite est instable, tandis que
  l'implicite est inconditionnellement stable. ({\bf 4 POINTS})
\end{enumerate}
\vspace{22cm}
\noindent {\bf Probl\`eme 2}. On consid\`ere l'\'equation de la chaleur (\ref{eqn:chaleur}), qu'on se propose de r\'esoudre en utilisant le sch\'ema num\'erique suivant (o\`u $\theta \ge 0$ est un nombre r\'eel positif) :

\begin{equation} \label{eqn:schema4}
\displaystyle (1+\theta) \, \frac{u_j^{n+1} \, - \, u_j^n}{\Delta t}
\, - \,\theta\, \frac{u_j^n \, - \, u_j^{n-1}}{\Delta t} \, - \nu \,  \frac{u_{j+1}^{n+1} \,- \,  2 \, u_j^{n+1} \, + \, u_j^{n+1}}{\Delta x ^2 } \,  =  \,   0 \qquad (j,n) \, \in \, \mathbb{Z}\times \mathbb{N}^* .
\end{equation}

\begin{enumerate}
\item Montrer, en calculant avec l'erreur de troncature,
  que le sch\'ema est en g\'en\'eral d'ordre $1$ en temps et $2$ en
  espace. Calculer la valeur de $\theta \, \ge \, 0$ pour laquelle le
  sch\'ema soit d'ordre $2$ en temps. ({\bf 2 POINTS})
\vspace{12cm}
\item Montrer que le sch\'ema (\ref{eqn:schema4}) est
  inconditionnellement stable. ({\bf 3 POINTS})
\end{enumerate}
\newpage
\noindent {\bf Probl\`eme 3} - On consid\`ere l'\'equation d'advection : \\

\begin{equation} \label{eqn:transport}
\left\{ 
\begin{array}{rcl}
\displaystyle \frac{\partial u}{\partial t} \, + \, c \, \frac{\partial u}{\partial x} & = & 0 \qquad (x,t) \in \mathbb{R} \times \mathbb{R}^*_+ \\
u(x,0) &= &\displaystyle u^0(x) \qquad x \in \mathbb{R}
\end{array}
\right.
\end{equation}

\begin{enumerate}
\item Donner le sch\'ema explicite d\'ecentr\'e amont dans le cas o\`u $c<0$ et
  $c>0$. ({\bf 1 POINT})
\vspace{3cm}
\item  En calculant le facteur d'amplification du sch\'ema dans ses
  deux versions, dire sous quelle condition il est $\displaystyle
  L^2$-stable. ({\bf 4 POINTS})
\newpage
\item On suppose maintenant que $c>0$ et que la  condition de stabilit\'e ci-dessus est v\'erifi\'ee.  Montrer que dans ce cas, on a aussi:
\begin{equation} \label{eqn:stab}
\sup_{j \, \in \, \mathbb{Z}} \left|  u_j^{n+1} \right| \; \le \, \sup_{j \, \in \, \mathbb{Z}} \left|  u_j^{n} \right| 
\end{equation}
Comment interpr\'eter l'in\'egalit\'e (\ref{eqn:stab})  ? ({\bf 2 POINTS})
\end{enumerate}
\vspace{9cm}
\noindent {\bf Probl\`eme 4}. On veut implementer num\'eriquement \`a
l'aide du logiciel \texttt{Scilab} le {sch\'ema} de Crank-Nicolson
$$
\left\{\begin{array}{l}
\displaystyle\frac{u_j^{n+1}-u_j^n}{\Delta t}-\nu
\frac{u_{j+1}^{n}-2u_j^n+u_{j-1}^{n}}{2\Delta x^2}-\nu
\frac{u_{j+1}^{n+1}-2u_j^{n+1}+u_{j-1}^{n+1}}{2\Delta x^2}=\frac{1}{2}(f^n_j+f^{n+1}_j), 1\le j
\le N,\\[2ex]
\displaystyle u^n_0=u^n_{N+1}=0,\, u_j^0=u_0(x_j), f^n_j = f(x_j^n),\, 1\le j \le N.
\end{array}\right.
$$
o\`u $u_0$ est la condition initiale et $f$ est le second membre.  On notera le vecteur de
inconnues par $U^n=(u^n_j)_{1\le
  j\le n}$, celui qui donnera le second membre par $F^n=(f^n_j)_{1\le
  j\le n}$ et le nombre de CFL par $\sigma=\frac{\nu\Delta t}{\Delta x^2}$.
\begin{enumerate}
\item Montrer que le sch\'ema peut s'\'ecrire sous forme matricielle
  ou l'on pr\'ecisera $A$. ({\bf 1 POINT})
$$
(I+\sigma/2 A)U^{n+1}= (I-\sigma/2 A)U^{n}+1/2
F^n+1/2 F^{n+1}.
$$
\newpage
\'Ecrire une fonction \texttt{Scilab} qui a l'en-t\^ete
  \texttt{function Un=heat(xspan,tspan,nu,u0,f)} qui
  calculera la solution de l'\'equation de la chaleur par le sch\'ema
  de Crank-Nicolson. Ses param\`etres sont: \texttt{xspan} (le vecteur des $x_j$),
  \texttt{tspan} (le vecteur des $t_n$), \texttt{nu} (le coefficient de
  diffusion $\nu$), la condition initiale \texttt{u0} ($u_0$) et le
  second membre $f$. (remarquons que le nombre d'inconnues en espace, $N$ peut
  s'obtenir comme \texttt{length(xspan)-2}). ({\bf 3 POINTS})
\vspace{10cm}

\item Consid\'erons maintenant le cas concret o\`u $(a,b) = (0,\pi)$,
  $\nu= 1$, $f (x, t) = -\sin(x) \sin(t) + \sin(x)\cos(t)$, condition
  initiale $u(x, 0) = \sin(x)$. Dans ce cas, la solution exacte est
  $u(x,t) = \sin(x)\cos(t)$. \'Ecrire un programme qui r\'esout ce probl\`eme sur
  l'intervalle en temps $[0,1]$ et tracer la solution exacte au temps
  final sur le m\^eme graphique que celle approch\'ee. ({\bf 3 POINTS})
\end{enumerate}

\end{document}

