\documentclass[12pt,a4paper]{article}

%\usepackage[T1]{fontenc} % Pour la bonne cesure du francais
\usepackage{amsmath} % Pour les symboles complementaire comme les matrices !
\usepackage{amssymb}
\usepackage{verbatim}
\usepackage{epsfig}
%\usepackage{/home/cohen/fortran/graphics/GGGraphics/GGGraphics}
%\usepackage{D:/GGGraphics/GGGraphics}

\newtheorem{theorem}{Theorem}
\newtheorem{corollary}[theorem]{Corollary}
\newtheorem{example}{Example}
\newtheorem{proposition}{Proposition}
\newtheorem{rem}{\noindent\textbf{\textit {Remarque\,}}}
\newcommand{\qed}{\hfill$\qedsquare$\goodbreak\bigskip}

\def\e{{\mathchoice{\hbox{\mathbb{R}m e}}{\hbox{\mathbb{R}m e}}%
        {\hbox{\mathbb{R}m \scriptsize e}}{\hbox{\mathbb{R}m \tiny e}}}}
        
\advance\voffset by -35mm \advance\hoffset by -25mm
\setlength{\textwidth}{175mm} \setlength{\textheight}{260mm}
\pagestyle{empty}

\begin{document}

\noindent {\large Universit\'e Côte d'Azur} \hfill Polytech Nice Sophia\\
\noindent MAM4 - \'Equations aux D\'eriv\'ees Partielles \hfill 
25 Octobre 2021 \\

\hrule

\vspace{1cm}
%\bigskip
%\parskip 6pt
\centerline {\large \sc Contrôle \'Equations aux D\'eriv\'ees Partielles. Dur\'ee : 50 minutes}
\vspace{1cm}


\hrule
\vspace{1cm}
\noindent {\sl  Les documents de cours ne sont pas autorisés. Justifier vos reponses d'une fa\c{c}on concise
et claire.}\\

\noindent {\bf Questions théoriques}:
\begin{itemize}
\item De quelle classe d'EDP fait partie l'équation d'advection?
\item Définir la diffusion numérique et donner un exemple de schéma diffusif et un non-diffusif.  Pourquoi cette notion est importante dans l'étude des schémas appliqués à l'equation d'advection?
\item La plupart des schémas diffusifs sont d'ordre $1$ en espace. Pourriez-vous donner un exemple d'un schéma d'ordre 2 en espace qui est diffusif?
\item Expliquer quelques différences fondamentales entre le schéma décentré Euler explicite et Lax-Wendroff.
\end{itemize}

\vspace{1cm}
\noindent On considère l'\'equation d'advection dans le domaine born\'e $(0,1)$:
$$
\begin{cases}
\displaystyle\frac{\partial u}{\partial t}+V\frac{\partial u}{\partial
  x}=0,\, \forall (x,t)\in(0,1)\times\mathbb{R}^+_*,
\end{cases}
$$
avec $u(x, 0) = u_0$, $u$ et $u_0$ p\'eriodiques de p\'eriode 1. \\

\noindent Afin de simplifier le raisonnement et fixer les idées on supposera que la vitesse d'advection $V$ est positive. (si $V$ est négatif on pourra faire un raisonnement symétrique).\\

\noindent {\bf Exercice} 

%\begin{enumerate}
%\item Montrer que le sch\'ema de {\it Lax-Friedrichs} 
%$$
%\frac{2u_j^{n+1}-u_{j+1}^{n}-u_{j-1}^{n}}{2\Delta t}+V \frac{u_{j+1}^{n}-u_{j-1}^{n}}{2\Delta x}=0.
%$$
%est stable en norme $L^2$ si $|V|\Delta x \le \Delta x$. \\

%Calculer l'erreur de troncature du sch\'ema.
%En d\'eduire que si le rapport $\Delta t/\Delta x$ est gard\'e
%constant quand $\Delta t$ et $\Delta x$ tendent vers $0$, alors le
%sch\'ema est consistant avec l'\'equation d'advection et pr\'ecis \`a l'ordre $1$
%et espace et en temps. 

%\item  
\noindent On va discrétiser l'équation par le schéma de {\it Lax-Wendroff} 
$$
\frac{u_j^{n+1}-u_{j}^{n}}{\Delta t}+V
\frac{u_{j+1}^{n}-u_{j-1}^{n}}{2\Delta x}-\frac{V^2\Delta t}{2}\frac{u_{j+1}^{n}-2u_j^n+u_{j-1}^{n}}{\Delta x^2}=0.
$$
\begin{itemize}
\item[a)]
En utilisant la méthode de von Neumann montrer que ce sch\'ema est $L^2$-stable sous la même condition CFL que le schéma explicite décentré $V\Delta t \le \Delta x$. \\ %(on va distinguer les cas $V>0$ et $V<0$).

\item[b)] En calculant l'erreur de troncature et en éffectuant les développements de Taylor autour des points bien choisis, montrer \'egalement qu'il est consistant avec l'\'equation d'advection et pr\'ecis \`a l'ordre $2$ et espace et en temps. 

\end{itemize}
%\end{enumerate}


\end{document}

