\documentclass[12pt,a4paper]{article}

%\usepackage[T1]{fontenc} % Pour la bonne cesure du francais
\usepackage{amsmath} % Pour les symboles complementaire comme les matrices !
\usepackage{amssymb}
\usepackage{verbatim}
\usepackage{epsfig}
%\usepackage{/home/cohen/fortran/graphics/GGGraphics/GGGraphics}
%\usepackage{D:/GGGraphics/GGGraphics}

\newtheorem{theorem}{Theorem}
\newtheorem{corollary}[theorem]{Corollary}
\newtheorem{example}{Example}
\newtheorem{proposition}{Proposition}
\newtheorem{rem}{\noindent\textbf{\textit {Remarque\,}}}
\newcommand{\qed}{\hfill$\qedsquare$\goodbreak\bigskip}

\def\e{{\mathchoice{\hbox{\mathbb{R}m e}}{\hbox{\mathbb{R}m e}}%
        {\hbox{\mathbb{R}m \scriptsize e}}{\hbox{\mathbb{R}m \tiny e}}}}
        
\advance\voffset by -35mm \advance\hoffset by -25mm
\setlength{\textwidth}{175mm} \setlength{\textheight}{260mm}
\pagestyle{empty}

\begin{document}

\noindent {\large Universit\'e Côte d'Azur} \hfill Polytech'Nice\\
\noindent MAM4 - \'Equations aux D\'eriv\'ees Partielles \hfill 
11 Janvier 2021 \\

\hrule

\vspace{0.6cm}
%\bigskip
%\parskip 6pt
\centerline {\large \sc Examen \'Equations aux D\'eriv\'ees Partielles. Dur\'ee : 2h}
\vspace{0.6cm}


\hrule
\vspace{0.7cm}
\noindent {\sl  Les documents de cours ne sont pas autorisés. Justifier vos reponses d'une fa\c{c}on concise
et claire.}\\

\noindent {\bf Probl\`eme 1}\\% ({\bf Total: 10 points}) \\

\noindent Consid\'erons l'\'equation d'advection dans le domaine born\'e $(0,1)$:
$$
\begin{cases}
\displaystyle\frac{\partial u}{\partial t}+V\frac{\partial u}{\partial
  x}=0,\, \forall (x,t)\in(0,1)\times\mathbb{R}^+_*,
\end{cases}
$$
avec $u(x, 0) = u_0$, $u$ et $u_0$ p\'eriodiques de p\'eriode 1.
\begin{enumerate}
\item Montrer que le sch\'ema de {\it Lax-Friedrichs} 
$$
\frac{2u_j^{n+1}-u_{j+1}^{n}-u_{j-1}^{n}}{2\Delta t}+V \frac{u_{j+1}^{n}-u_{j-1}^{n}}{2\Delta x}=0.
$$
est stable en norme $L^2$ si $|V|\Delta x \le \Delta x$. \\

Calculer l'erreur de troncature du sch\'ema.
En d\'eduire que si le rapport $\Delta t/\Delta x$ est gard\'e
constant quand $\Delta t$ et $\Delta x$ tendent vers $0$, alors le
sch\'ema est consistant avec l'\'equation d'advection et pr\'ecis \`a l'ordre $1$
et espace et en temps. 

\item  Montrer que le sch\'ema de {\it Lax-Wendroff} ne pr\'eserve pas le principe du maximum
discret
$$
\frac{u_j^{n+1}-u_{j}^{n}}{\Delta t}+V
\frac{u_{j+1}^{n}-u_{j-1}^{n}}{2\Delta x}-\frac{V^2\Delta t}{2}\frac{u_{j+1}^{n}-2u_j^n+u_{j-1}^{n}}{\Delta x^2}=0.
$$
sauf si le rapport $V\Delta t/\Delta x$ vaut $-1$, $0$ ou $1$. \\

Montrer que ce sch\'ema  est $L^2$-stable sous la condition CFL $|V|\Delta t \le
\Delta x$. \\ %(on va distinguer les cas $V>0$ et $V<0$).

Montrer \'egalement qu'il est consistant avec l'\'equation d'advection et pr\'ecis \`a l'ordre $2$
et espace et en temps. 
\end{enumerate}

\noindent {\bf Probl\`eme 2} \\ %({\bf Total: 10 points}) \\

\noindent Consid\'erons l'\'equation d'advection-diffusion dans le domaine born\'e $(0,1)$:
$$
\begin{cases}
\displaystyle\frac{\partial u}{\partial t}+V\frac{\partial u}{\partial
  x}-\nu\frac{\partial^2u}{\partial x^2}=0,\, \forall (x,t)\in(0,1)\times\mathbb{R}^+_*,
\end{cases}
$$
avec $u(x, 0) = u_0$, $u$ et $u_0$ p\'eriodiques de p\'eriode 1.\\
Consid\'erons le sch\'ema d\'ecentr\'e amont suivant:
$$
\frac{u_j^{n+1}-u_{j}^{n}}{\Delta t}+V
\frac{u_{j}^{n}-u_{j-1}^{n}}{\Delta x}-\nu\frac{u_{j+1}^{n}-2u_j^n+u_{j-1}^{n}}{\Delta x^2}=0.
$$
\begin{enumerate}
\item D\'eterminer l'ordre du sch\'ema. 
\item On veut d\'eterminer les conditions des stabilit\'e $L^2$ du sch\'ema lorsque $V > 0$ et
$V < 0$. Pour cela on proc\'edera en plusieurs \'etapes. \'Ecrire
d'abord le facteur d'amplification $G(k)$ sous la forme
$$
G(k)=\alpha e^{2i\pi k\Delta x}+\beta +\gamma  e^{-2i\pi k\Delta x},\, \alpha+\beta+\gamma=1.
$$
avec des $\alpha$, $\beta$, $\gamma$ que l'on pr\'ecisera. \\

\noindent Calculer le module complexe de $G(k)$ et montrer qu'il peut se mettre
sous la forme
$$
|G(k)|^2=(1-2(\alpha+\gamma)s_k)^2+4(\alpha-\gamma)^2s_k(1-s_k),\, s_k=\sin^2(k\pi\Delta x)
$$
(sans remplacer pour le moment les valeurs de coefficients). \\

\noindent En d\'eduire que la condition de stabilit\'e
$|G(k)|^2 \le 1$ est satisfaite si $(\alpha-\gamma)^2\le (\alpha+\gamma)$.
Remplacer maintenant les coefficients $\alpha$ et $\gamma$ et donner
la condition de stabilit\'e en fonction des param\`etres du
probl\`eme. Dans le cas o\`u $V < 0$ que se passe-t-il si $\nu\rightarrow 0$? 
\end{enumerate}

\noindent {\bf Probl\`eme 3} \\ 

\noindent  On cherche \`a r\'esoudre le probl\`eme aux
limites le suivant dans le carr\'e $\Omega=[-1,1]^2$, 
\begin{equation}\label{eq:tube}
\left\{\begin{array}{lcl}
-\Delta u &=& f,\,\text{dans }\Omega\\[2ex]
\displaystyle \frac{\partial u}{\partial \mathbf{n}}(x,\pm 1)&=&1,\,\forall x\in (-1,1),\\[2ex]
u(\pm 1,y)&=&0,\,\forall y\in  (-1,1).
\end{array}\right.
\end{equation}
où $\mathbf{n}$ est la normale extérieure à la frontière du domaine. 
\begin{enumerate}
\item En multipliant par une fonction test $v$ et en intégrant par parties, d\'eterminer la formulation variationnelle (FV) de ce problème. On va préciser l'espace $X$ sur lequel cette formulation est définie ainsi que la forme bilinéaire $a$ et la forme linéaire.
\item Montrer que $u$ est solution de (FV) ssi elle minimise sur $X$ une fonctionnelle $E(v)$ que l'on pr\'ecisera.
%\item Peut-on \'etablir l'unicit\'e de la solution \`a partir de cette \'equivalence ? (on suppose qu'il existent deux solutions. etc...)
\end{enumerate}
 \texttt{Indication}: apr\`es la
  multiplication par la fonction test $v$ et integration par parties,
  on va constater que l'integrale sur la fronti\`ere $\partial\Omega$
  se d\'ecompose en $4$ parties correspondant aux cot\'es du carr\'e
  et o\`u les conditions aux limites sont diff\'erentes.\\


\end{document}

