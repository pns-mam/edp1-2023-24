\documentclass[12pt,a4paper]{article}

%\usepackage[T1]{fontenc} % Pour la bonne cesure du francais
\usepackage{amsmath} % Pour les symboles complementaire comme les matrices !
\usepackage{amssymb}
\usepackage{verbatim}
\usepackage{epsfig}
%\usepackage{/home/cohen/fortran/graphics/GGGraphics/GGGraphics}
%\usepackage{D:/GGGraphics/GGGraphics}

\newtheorem{theorem}{Theorem}
\newtheorem{corollary}[theorem]{Corollary}
\newtheorem{example}{Example}
\newtheorem{proposition}{Proposition}
\newtheorem{rem}{\noindent\textbf{\textit {Remarque\,}}}
\newcommand{\qed}{\hfill$\qedsquare$\goodbreak\bigskip}

\def\e{{\mathchoice{\hbox{\mathbb{R}m e}}{\hbox{\mathbb{R}m e}}%
        {\hbox{\mathbb{R}m \scriptsize e}}{\hbox{\mathbb{R}m \tiny e}}}}
        
\advance\voffset by -35mm \advance\hoffset by -25mm
\setlength{\textwidth}{175mm} \setlength{\textheight}{260mm}
\pagestyle{empty}

\begin{document}

\noindent {\large Universit\'e Côte d'Azur} \hfill Polytech Nice Sophia\\
\noindent MAM4 - \'Equations aux D\'eriv\'ees Partielles \hfill 
2 Décembre 2022 \\

\hrule

\vspace{0.5cm}
%\bigskip
%\parskip 6pt
\centerline {\large \sc Examen \'Equations aux D\'eriv\'ees Partielles. Dur\'ee : 1H30}
\vspace{0.5cm}


\hrule
\vspace{0.7cm}
\noindent {\sl  Les documents de cours ne sont pas autorisés. Justifier vos réponses d'une fa\c{c}on concise
et claire.}\\

\noindent {\bf Questions de cours} :
\begin{itemize}
\item[--] Donner la définition des espaces de Hilbert $L^2(\Omega)$ et $H^1(\Omega)$, des produits scalaires associés et des normes induites.
\item[--] Énoncer le théorème de Lax-Milgram. 
\item[--] Donner un exemple et ensuite expliquer la différence entre les conditions aux limites essentielles et naturelles.
\end{itemize}

\noindent {\bf Exercice 1}\\
\noindent Consid\'erons l'\'equation d'advection dans le domaine born\'e $(0,1)$:
$$
\begin{cases}
\displaystyle\frac{\partial u}{\partial t}-4\frac{\partial u}{\partial
  x}=0,\, \forall (x,t)\in(0,1)\times\mathbb{R}^+_*,
\end{cases}
$$
avec $u(x, 0) = u_0$, $u$ et $u_0$ p\'eriodiques de p\'eriode 1.
\begin{enumerate}
\item Construire un schéma décentré explicite, stable en norme $L^2$ et en norme $L^{\infty}$ et donner sa condition de stabilité. Avec une petite modification le schéma devient {\it inconditionnellement instable}. Donner ce nouveau schéma et montrer son instabilité. 

\item Calculer l'erreur de troncature dans les deux cas, quel est l'ordre d'approximation?

\item Montrer que le schéma est dissipatif et donner son equation équivalente.

\item  On souhaite améliorer la stabilité, l'ordre d'approximation et la dissipation du schéma.

Construire un schéma (i) inconditionnellement stable sans changer l'ordre d'approximation. (ii) inconditionnellement stable mais avec ordre 2 en temps et espace. (iii) non-dissipatif en utilisant l'équation équivalente.

\end{enumerate}

\noindent {\bf Exercice 2} \\ 
Considérons le problème aux limites suivant:
 \begin{equation}\label{eq:bvp2}
\left\{\begin{array}{rl}
-((x^2+1)u'(x))' + u(x) &= f(x), \quad x\in (0,1)\\
 u'(0)+u(0) &= 2,\\
 u'(1) &= 1,
 \end{array}\right.
 \end{equation}
 où $f$ est une fonction à carré integrable.\\
En multipliant l'équation (\ref{eq:bvp2}) par une fonction test $v$, en integrant par parties et en prenant en compte les conditions aux limites, écrire la formulation variationnelle du problème (\ref{eq:bvp2}) sous la forme
$$
\mbox{Trouver } u\in V_E, \mbox{tel que } a(u,v) = (f,v),\, \forall v\in V_E.
$$
Identifier clairement la forme bilinéaire $a$ et l'espace fonctionnel $V_E$.

\end{document}

