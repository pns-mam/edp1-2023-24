\documentclass[12pt,a4paper]{article}

%\usepackage[T1]{fontenc} % Pour la bonne cesure du francais
\usepackage{amsmath} % Pour les symboles complementaire comme les matrices !
\usepackage{amssymb}
\usepackage{verbatim}
\usepackage{epsfig}
%\usepackage{/home/cohen/fortran/graphics/GGGraphics/GGGraphics}
%\usepackage{D:/GGGraphics/GGGraphics}

\newtheorem{theorem}{Theorem}
\newtheorem{corollary}[theorem]{Corollary}
\newtheorem{example}{Example}
\newtheorem{proposition}{Proposition}
\newtheorem{rem}{\noindent\textbf{\textit {Remarque\,}}}
\newcommand{\qed}{\hfill$\qedsquare$\goodbreak\bigskip}

\def\e{{\mathchoice{\hbox{\mathbb{R}m e}}{\hbox{\mathbb{R}m e}}%
        {\hbox{\mathbb{R}m \scriptsize e}}{\hbox{\mathbb{R}m \tiny e}}}}
        
\advance\voffset by -35mm \advance\hoffset by -25mm
\setlength{\textwidth}{175mm} \setlength{\textheight}{260mm}
\pagestyle{empty}

\begin{document}

\noindent {\large Universit\'e Côte d'Azur} \hfill Polytech Nice Sophia\\
\noindent MAM4 - \'Equations aux D\'eriv\'ees Partielles \hfill 
20 Octobre 2022 \\

\hrule

\vspace{0.8cm}
%\bigskip
%\parskip 6pt
\centerline {\large \sc Contrôle EDP (Sujet B) Dur\'ee : 90 minutes}
\vspace{0.8cm}


\hrule
\vspace{0.5cm}
\noindent {\sl  Les documents de cours ne sont pas autorisés. Justifier vos réponses d'une fa\c{c}on concise
et claire.}\\

\noindent {\bf Questions théoriques}:
\begin{itemize}
\item Définir la notion de stabilité en norme $L^{2}$ et expliquer brièvement comment on peut prouver qu'un schéma est stable par rapport à cette norme.
\item Écrire le schéma d'Euler implicite centré pour l'équation d'advection. 
\item Donner l'expression de l'erreur de consistance pour le schéma d'Euler explicite appliqué à l'équation de la chaleur.
\end{itemize}

\vspace{0.5 cm}


\noindent {\bf Exercice 1}

\noindent On consid\`ere l'\'equation de la chaleur :
\begin{equation} \label{eqn:chaleur}
\left\{
\begin{array}{rcl}
\displaystyle \frac{\partial u}{\partial t} \, - \nu\, \frac{\partial^2 u}{\partial x^2}  & =  & 0  \qquad \forall \, (x,t) \, \in \, \mathbb{R} \times \mathbb{R}^+ \\
\displaystyle u(x,0)  & =  & u_0(x)  \qquad \forall \, x \, \in \, \mathbb{R}
\end{array}
\right.
\end{equation}
et on se propose de la r\'esoudre en utilisant le sch\'ema num\'erique explicite suivant
\begin{equation} \label{eqn:schema2}
\displaystyle  \frac{u_j^{n+1} \, - \, u_j^{n-1}}{2 \Delta t} \, - \nu\,  \frac{u_{j+1}^{n} \,- \,  2 \, u_j^{n} \, + \, u_{j-1}^{n}}{\Delta x ^2 } \,  =  \,   0 \qquad (j,n) \, \in \, \mathbb{Z} \times \mathbb{N}^* .
\end{equation}

\begin{enumerate}
\item Montrer que le schéma (\ref{eqn:schema2}) est consistant et donner son ordre d'approximation.
\item Montrer que ce sch\'ema explicite est inconditionnellement instable. 
\end{enumerate}
\texttt{Indication}: Si le facteur d'amplification vérifie une équation quadratique, le schéma est stable si les deux solutions sont inférieures à $1$ en module. Il est instable si au moins une des solutions est strictement supérieure à $1$ en module.\\

\noindent {\bf Exercice 2}

\noindent On considère l'\'equation d'advection dans le domaine born\'e $(0,1)$ avec $V>0$:
$$
\begin{cases}
\displaystyle\frac{\partial u}{\partial t}+V\frac{\partial u}{\partial
  x}=0,\, \forall (x,t)\in(0,1)\times\mathbb{R}^+_*,
\end{cases}
$$
avec $u(x, 0) = u_0$, $u$ et $u_0$ p\'eriodiques de p\'eriode 1. \\

\begin{enumerate}
\item Montrer que le sch\'ema de {\it Lax-Friedrichs} implicite
$$
\frac{2u_j^{n+1}-u_{j+1}^{n}-u_{j-1}^{n}}{2\Delta t}+V \frac{u_{j+1}^{n+1}-u_{j-1}^{n+1}}{2\Delta x}=0.
$$
est inconditionnellement stable en norme $L^2$. \\

\item Calculer l'erreur de troncature du sch\'ema.
En d\'eduire que si le rapport $\Delta t/\Delta x$ est gard\'e
constant quand $\Delta t$ et $\Delta x$ tendent vers $0$, alors le
sch\'ema est consistant avec l'\'equation d'advection. Quelle est sa précision? 

%\item  
%\noindent On va discrétiser l'équation par le schéma de {\it Lax-Wendroff} 
%$$
%\frac{u_j^{n+1}-u_{j}^{n}}{\Delta t}+V
%\frac{u_{j+1}^{n}-u_{j-1}^{n}}{2\Delta x}-\frac{V^2\Delta t}{2}\frac{u_{j+1}^{n}-2u_j^n+u_{j-1}^{n}}{\Delta x^2}=0.
%$$
%\begin{itemize}
%\item[a)]
%En utilisant la méthode de von Neumann montrer que ce sch\'ema est $L^2$-stable sous la même condition CFL que le schéma explicite décentré $V\Delta t \le \Delta x$. \\ %(on va distinguer les cas $V>0$ et $V<0$).
%
%\item[b)] En calculant l'erreur de troncature et en éffectuant les développements de Taylor autour des points bien choisis, montrer \'egalement qu'il est consistant avec l'\'equation d'advection et pr\'ecis \`a l'ordre $2$ et espace et en temps. 
%
%\end{itemize}
\end{enumerate}




\end{document}

