\documentclass[12pt,a4paper]{article}

%\usepackage[T1]{fontenc} % Pour la bonne cesure du francais
\usepackage{amsmath} % Pour les symboles complementaire comme les matrices !
\usepackage{amssymb}
\usepackage{verbatim}
\usepackage{epsfig}
%\usepackage{/home/cohen/fortran/graphics/GGGraphics/GGGraphics}
%\usepackage{D:/GGGraphics/GGGraphics}

\newtheorem{theorem}{Theorem}
\newtheorem{corollary}[theorem]{Corollary}
\newtheorem{example}{Example}
\newtheorem{proposition}{Proposition}
\newtheorem{rem}{\noindent\textbf{\textit {Remarque\,}}}
\newcommand{\qed}{\hfill$\qedsquare$\goodbreak\bigskip}

\def\e{{\mathchoice{\hbox{\mathbb{R}m e}}{\hbox{\mathbb{R}m e}}%
        {\hbox{\mathbb{R}m \scriptsize e}}{\hbox{\mathbb{R}m \tiny e}}}}
        
\advance\voffset by -35mm \advance\hoffset by -25mm
\setlength{\textwidth}{175mm} \setlength{\textheight}{260mm}
\pagestyle{empty}

\begin{document}

\noindent {\large Universit\'e Côte d'Azur} \hfill Polytech'Nice\\
\noindent MAM4 - \'Equations aux D\'eriv\'ees Partielles \hfill 
28 Janvier 2021 \\

\hrule

\vspace{0.6cm}
%\bigskip
%\parskip 6pt
\centerline {\large \sc Examen \'Equations aux D\'eriv\'ees Partielles - Sujet B. Dur\'ee : 1,5h}
\vspace{0.6cm}


\hrule
\vspace{0.7cm}
\noindent {\sl  Les documents de cours sont autorisés. Justifier vos reponses d'une fa\c{c}on concise
et claire.}\\

\noindent {\bf Probl\`eme 1}\\% ({\bf Total: 10 points}) \\
\noindent Consid\'erons l'\'equation d'advection dans le domaine born\'e $(0,1)$:
$$
\begin{cases}
\displaystyle\frac{\partial u}{\partial t}-4\frac{\partial u}{\partial
  x}=0,\, \forall (x,t)\in(0,1)\times\mathbb{R}^+_*,
\end{cases}
$$
avec $u(x, 0) = u_0$, $u$ et $u_0$ p\'eriodiques de p\'eriode 1.
\begin{enumerate}
\item Construire un schéma décentré explicite stable en norme $L^2$ et en norme $L^{\infty}$ et donner sa condition de stabilité. Avec une petite modification le schéma devient inconditionnellement instable. Donner ce nouveau schéma et montrer son instabilité. 

\item Calculer l'erreur de troncature dans les deux cas, quel est l'ordre d'approximation?

\item  On souhaite améliorer la stabilité, l'ordre d'approximation et la dissipation du schéma.

Construire un schéma (i) inconditionnellement stable sans changer l'ordre d'approximation. (ii) inconditionnellement stable mais avec ordre 2 en temps et espace, (iii) non-dissipatif en passant par l'equation equivalente.


\end{enumerate}

\noindent {\bf Probl\`eme 2} \\ 
 Montrer que la fonction $f: (0,1)\rightarrow \mathbb{R}$ donnée par
 \begin{equation}\label{eq:rhs}
 f(x) = \left\{\begin{array}{rl}
 6x,& 0 <  x\le 1/2,\\
 2, &1/2 <x < 1
 \end{array}\right.
 \end{equation}
est à carré integrable mais pas continue sur $(0,1)$.\\
%\marks{2}
Considérons le problème aux limites suivant:
 \begin{equation}\label{eq:bvp}
\left\{\begin{array}{rl}
-u''(x) &= f(x), \quad x\in (0,1)\\
 u(0) &= u(1) = 0,
 \end{array}\right.
 \end{equation}
 où $f(x)$ est donné par (\ref{eq:rhs}). 
Montrer que la fonction suivante est une solution faible du problème aux limites (\ref{eq:bvp}) 
\begin{equation}\label{eq:sol}
 u(x) = \left\{\begin{array}{rl}
 -x^3 +\frac{3x}{4}, & 0 \le  x\le 1/2,\\
 -x^2+x, & 1/2 <x \le 1.
  \end{array}\right.
 \end{equation}

\noindent {\bf Probl\`eme 3} \\ 
Considérons le problème aux limites suivant:
 \begin{equation}\label{eq:bvp2}
\left\{\begin{array}{rl}
-(xu'(x))' + u(x) &= f(x), \quad x\in (0,1)\\
 u'(0)+u(0) &= 0,\\
 u'(1) &= 0,
 \end{array}\right.
 \end{equation}
 où $f$ est une fonction à carré integrable.\\
En multipliant l'équation (\ref{eq:bvp2}) par une fonction test $v$, en integrant par parties et en prenant en compte les conditions aux limites, écrire la formulation variationnelle du problème (\ref{eq:bvp2}) sous la forme
$$
\mbox{Trouver } u\in V_E, \mbox{tel que } a(u,v) = (f,v),\, \forall v\in V_E.
$$
Identifier clairement la forme bilinéaire $a$ et l'espace fonctionnel $V_E$ .

\end{document}

