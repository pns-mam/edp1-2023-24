\documentclass[12pt,a4paper]{article}

%\usepackage[T1]{fontenc} % Pour la bonne cesure du francais
\usepackage{amsmath} % Pour les symboles complementaire comme les matrices !
\usepackage{amssymb}
\usepackage{verbatim}
\usepackage{epsfig}
%\usepackage{/home/cohen/fortran/graphics/GGGraphics/GGGraphics}
%\usepackage{D:/GGGraphics/GGGraphics}

\newtheorem{theorem}{Theorem}
\newtheorem{corollary}[theorem]{Corollary}
\newtheorem{example}{Example}
\newtheorem{proposition}{Proposition}
\newtheorem{rem}{\noindent\textbf{\textit {Remarque\,}}}
\newcommand{\qed}{\hfill$\qedsquare$\goodbreak\bigskip}

\def\e{{\mathchoice{\hbox{\mathbb{R}m e}}{\hbox{\mathbb{R}m e}}%
        {\hbox{\mathbb{R}m \scriptsize e}}{\hbox{\mathbb{R}m \tiny e}}}}
        
\advance\voffset by -35mm \advance\hoffset by -25mm
\setlength{\textwidth}{175mm} \setlength{\textheight}{260mm}
\pagestyle{empty}

\begin{document}

\noindent {\large Universit\'e Côte d'Azur} \hfill Polytech Nice Sophia \\
\noindent MAM4 - \'Equations aux D\'eriv\'ees Partielles \hfill 
29 Novembre 2021 \\

\hrule

\vspace{1cm}
%\bigskip
%\parskip 6pt
\centerline {\large \sc Contrôle \'Equations aux D\'eriv\'ees Partielles. Dur\'ee : 50 minutes}
\vspace{1cm}


\hrule
\vspace{1cm}
\noindent {\sl  Les documents de cours ne sont pas autorisés. Justifier vos reponses d'une fa\c{c}on concise
et claire.}\\

\noindent {\bf Questions théoriques}:
\begin{itemize}
\item Quelles sont les classes principales d'EDP? Donner un exemple de chaque classe.
\item Quels sont les avantages des schémas explicites et implicites? Donner un exemple de chaque catégorie.
\item Définir la notion de stabilité pour une norme vectorielle générique. 
\item Décrire brièvement la méthode de von Neumann pour étudier la stabilité en norme $L^2$.
\end{itemize}

\noindent On consid\`ere le probl\`eme de la chaleur en une dimension
d'espace 
\begin{equation}\label{eq:heat}
\left\{\begin{array}{lcll}
\displaystyle\frac{\partial u}{\partial t} - \nu \frac{\partial^2 u}{\partial x^2}&=
&0,\,
(x,t)\in (0,1)\times\mathbb{R}_*^+, & \text{Équation à l'intérieur du domaine } \\
u(0,t)&=&u(1,t) = 0,\, t\in \mathbb{R}_*^+ &\text{CL Dirichlet}\\
u(x,0)&=&u_0(x),\, x\in (0,1) & \text {Condition initiale}.
\end{array}\right.
\end{equation}

%\noindent {\bf Exercice 1}\\% ({\bf Total: 10 points}) \\
%On admettra (sans demonstration) l'inégalité de Poincaré:  toute fonction $v(x)$ contin\^ument d\'erivable sur
%$[0,1]$ t.q. $v(0)=0$, v\'erifie $\int_0^1 v^2(x)dx \le \int_0^1\left(\frac{d v}{d
%    x}\right)^2dx.$
%Établir l'\'egalit\'e de l'énergie: %({\bf 2.5 points}) :
%$$
%\frac{1}{2}\frac{dE(t)}{dt}
%=-\nu\int_0^1\left(\frac{\partial u}{\partial x}\right)^2 dx, \, \text{où } E(t) = \int_0^1 u^2(x,t)dx,
%$$
%et (en appliquant l'inégalité de Poincaré) d\'eduire que l'\'energie {\it d\'ecro\^it exponentiellement en temps} %({\bf 1.5 points}) 
%\begin{equation} \label{eq:energ}
%E(t) \le E(0)e^{-2\nu t}.
%\end{equation}


\noindent {\bf Exercice} \\ %({\bf Total: 10 points}) \\

\noindent On discr\'etise le domaine en utilisant un maillage r\'egulier
$(t_n,x_j)=(n\Delta t,j\Delta x)$,  $\forall n\ge 0,j\in\{0,1,...,N+1\}$ o\`u $\Delta x=1/(N+1)$ et $\Delta t>0$. \\

\begin{itemize}
\item Montrer que le sch\'ema suivant
\begin{equation}\label{sch:1}
\frac{u_j^{n+1}-u_j^n}{\Delta t}-\frac{\nu}{2} \frac{u_{j+1}^{n}-2u_j^n+u_{j-1}^{n}}{\Delta x^2}-\frac{\nu}{2} \frac{u_{j+1}^{n+1}-2u_j^{n+1}+u_{j-1}^{n+1}}{\Delta x^2}=0.
\end{equation}
est consistant, d'ordre 2 en temps et en espace et inconditionnellement stable en norme $L_2$. 

\item Écrire ce schéma sous forme matricielle et expliquer exactement quel système linéaire il faut résoudre à chaque pas de temps.

\end{itemize}

\noindent \texttt{Indication}: Pour étudier la consistance on va écrire d'abord l'erreur de troncature. On va faire des développements de Taylor en espace et en temps de la solution jusqu'à un ordre suffisant permettant de déterminer la précision et simplifier les termes de l'erreur de troncature (e.g. 4 en espace, 2 ou 3 en temps dépendant du schéma)
%\noindent \texttt{Indication} Les schémas du type \eqref{sch:1} et \eqref{sch:2} font partie de la même catégorie. De ce fait la consistance et la stabilité de ces derniers peuvent être analysées en même temps.\\




\end{document}

