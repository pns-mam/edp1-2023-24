\documentclass[12pt,a4paper]{article}

\usepackage[T1]{fontenc} % Pour la bonne cesure du francais
\usepackage{amsmath} % Pour les symboles complementaire comme les matrices !
\usepackage{amssymb}
\usepackage{verbatim}
\usepackage{epsfig}
%\usepackage{/home/cohen/fortran/graphics/GGGraphics/GGGraphics}
%\usepackage{D:/GGGraphics/GGGraphics}

\newtheorem{theorem}{Theorem}
\newtheorem{corollary}[theorem]{Corollary}
\newtheorem{example}{Example}
\newtheorem{rem}{\noindent\textbf{\textit {Remarque\,}}}
\newcommand{\qed}{\hfill$\qedsquare$\goodbreak\bigskip}

\def\e{{\mathchoice{\hbox{\mathbb{R}m e}}{\hbox{\mathbb{R}m e}}%
        {\hbox{\mathbb{R}m \scriptsize e}}{\hbox{\mathbb{R}m \tiny e}}}}
        
\advance\voffset by -35mm \advance\hoffset by -25mm
\setlength{\textwidth}{175mm} \setlength{\textheight}{260mm}
\pagestyle{empty}

\begin{document}

\noindent {\large Universit\'e de Nice-Sophia Antipolis} \hfill EPU\\
\noindent Math\'ematiques Appliqu\'ees et Mod\'elisation (MAM4/SI4) \hfill mercredi 17 octobre 2012 \\

\hrule

\bigskip
\bigskip

\begin{center}{\bf \'Equations aux d\'eriv\'ees partielles -- TP Scilab}\end{center}

\bigskip

%\parskip 12pt
\noindent Le but de cette s\'eance est d'implementer num\'eriquement \`a l'aide
du logiciel \texttt{Scilab} quelques sch\'emas numeriques, de tester
leurs propri\'et\'es de stabilit\'e et repr\'esenter graphiquement
l'\'evolution de la solution au cours des it\'erations en temps.

\paragraph{Introduction.} On consid\`ere le probl\`eme mod\`ele 
$$
\left\{\begin{array}{l}
\displaystyle \frac{\partial u}{\partial t}(x,t)+a\frac{\partial
  u}{\partial x}(x,t)=0,x\in\mathbb{R},x\in[0,L], t\ge 0,\\
\displaystyle u(x,0)=u_0(x),\,x\in [0,L],\\
u(0,t)=u(L,t),\,t\ge 0.
\end{array}\right.
$$
Pour simplifier on a suppos\'ee que la donn\'ee initale
est p\'eriodique de p\'eriode $L$, si bien que nous pouvons nous
restreindre \`a l'intervalle $[0,L]$. On v\'erifie ais\'ement que la
solution exacte est donn\'ee par $u(x,t)=u_0(x-at)$.\\

\noindent On cherche \`a approcher num\'eriquement la solution par le sch\'ema {\it upwind} suivant:
$$
\frac{u_i^{n+1}-u_i^n}{\Delta t}+a\frac{u_i^n-u_{i-1}^n}{\Delta x}=0.
$$ 
o\`u $u_i^n\cong u(x_i,t^n),\,1\le i\le N$ et $\Delta x=L/(n-1)$,
$x_i=(i-1)\Delta x$ et $t^n = n\Delta t$. En r\'earrangeant les termes
on obtient:
$$
u_i^{n+1}=u_i^n-\sigma(u_i^n-u_{i-1}^n)
$$
o\`u $\sigma=\displaystyle\frac{a\Delta t}{\Delta x}$ est connu sous le nom de {\it nombre de Courant}.\\
Pour approcher la condition initiale et la condition limite on \'ecrit:
$$
u_i^0=u_0(x_i),\,0\le i\le N,\, u_1^n = u_N^n,\,n\ge 0.
$$
\paragraph{Premier script Scilab.} On \'ecrira un script Scilab \texttt{script1.sce} qui comprendra 4 parties:
\begin{enumerate}
\item Initialisation des param\`etres physiques: vitesse de propagation, longueur du domaine, temps maximum et condition initiale. La condition initiale sera une fonction en cr\'eneau:
$$
u_0(x)=\left\{\begin{array}{l} 1,\,1<x<1.5,\\0,\,\mbox{sinon}.\end{array}\right.
$$
\item Initialisation des param\`etres num\'eriques. On prendra $N=201$ et on d\'eduira $\Delta x$, puis on choisira $\sigma$ et on d\'eduira $\Delta t$.
\item Boucle en temps: initaliser $u_i^0,\, i=1,...,N$, puis \'evaluer
  $u_i^{n+1},\, i=1,..., N$ en fonction de $u_i^n,\, i=1,...,N$ pour $t^n<T$.
\item Visualisation graphique des r\'esultats: afficher sur la m\^eme fen\^etre la solution exacte et la solution approch\'ee.
\end{enumerate}
\paragraph{Indications.} Initialiser le script par la commande \texttt{clear} qui nettoie
l'espace m\'emoire occup\'e avant le lancement du script.
\begin{enumerate}
\item Param\`etres physiques: 
\begin{verbatim}
a = 0.1;   // vitesse d'advection
Tmax = 10; // temps maximum
L = 5;     // longueur du domaine
\end{verbatim}
\noindent Pour la condition initiale on utilisera la fonction \texttt{bool2s}:
\begin{verbatim}
function [v] = condinit(x) 
  v = bool2s((x>1.) & (x<1.5));
endfunction
\end{verbatim}
\item Param\`etres num\'eriques:
\begin{verbatim}
N = 201;      // nb de points de discretisation en espace 
dx = L/(N-1); // pas d'espace 
s = 0.8;      // nombre de Courant 
dt = s*dx/a;  // pas de temps
\end{verbatim}
\noindent Initialisation du maillage et de la donn\'ee initiale:
\begin{verbatim}
x = linspace(0,L,N)';
u = condinit(x);
\end{verbatim}
\item Boucle en temps: \`a chaque pas de temps on met \`a jour la
solution de fa\c{c}on vectorielle en tenant compte de la condition de
p\'eriodicit\'e:
\begin{verbatim}
tps = dt:dt:Tmax;
for t = tps
   uold = u; 
   u(2:N) = uold(2:N)-s*(uold(2:N)-uold(1:N-1));
   u(1)=u(N);
end
t = tps($); // $ pointe sur le dernier element du vecteur
\end{verbatim}
\item Sortie graphique: on pourra par exemple utiliser les commandes suivantes:
\begin{verbatim}
clf()
plot(x,u,'rx-',x,uexact,'bo-');
xtitle('Comparaison entre la solution exacte et approchee','x','Solution');
legend('Sol. approchee','Sol. exacte');
\end{verbatim}
\end{enumerate}

\noindent Pour executer le script taper dans la fen\^etre Scilab la
commande \texttt{ exec script1.sce}. \\ Afin de mesurer le temps
d'ex\'ecution du programme on ins\'erera la commande \texttt{timer()}
avant le d\'emarrage de la boucle en temps et en derni\`ere ligne du
script Scilab.\\

\noindent {\bf Question}: Augmenter progressivement le param\`etre $\sigma$ et
observer le r\'esultat. Quelle est la valeur critique?
\newpage
\paragraph{Deuxi\`eme script Scilab.} On souhaite \'egalement pouvoir consid\'erer la condition initiale suivante:
$$
u_0(x) = \sin(8\pi x/L)
$$ 
et avoir le choix entre $2$ sch\'emas num\'eriques: le sch\'ema
{\it upwind} pr\'ec\'edent et le sch\'ema de {\it Lax-Wendroff}:
$$
u_i^{n+1}=u_i^n-\frac{\sigma}{2}(u_{i+1}^n-u_{i-1}^n)+\frac{\sigma^2}{2}(u_{i+1}^n-2u_i^n+u_{i-1}^n)
$$
\'Ecrire un script Scilab qu'on appelera \texttt{script2.sce} et qui offre ces $2$ options.

\paragraph{Indications Scilab.} On introduira deux nouvelles variables qu'on pourra appeler
\texttt{condI} et \texttt{scheme} et on utilisera la commande
\texttt{select}:
\begin{verbatim}
function [v] = condinit(x)
    select condI
       case 1
         v = bool2s((x>1.) & (x<1.5));
       case 2
         v = sin(8*%pi*x/L);
    end
endfunction
\end{verbatim}
\noindent {\bf Question}: Repartir de $\sigma=0.8$ et essayer 4 possibilit\'es
du couple conditions initiale/sch\'ema num\'erique. Quelles conditions
tirez-vous?\\ 
Augmenter progressivement $\sigma$ pour le sch\'ema de
Lax-Wendroff et observer.

\paragraph{Troisi\`eme script Scilab: animation graphique.} L'objectif
de cette section est de pouvoir visualiser la solution num\'erique au cours de it\'erations au lieu de visualiser que la
solution finale. On r\'ealisera un nouveau script appel\'e
\texttt{script3.sce}

\paragraph{Indications Scilab}

\begin{enumerate}
\item Avant la boucle en temps on va initialiser/\'effacer la fen\^etre graphique:
\begin{verbatim}
clf()
\end{verbatim}

\item Au sein de la boucle en temps, la fen\^etre graphique est
rafraichie de la fa\c{c}on suivante: 
\begin{verbatim}
uexact =...;
plot(...); 
xtitle(...); 
legend(...)
halt('Press a key') 
clf()
\end{verbatim}
L commande \texttt{halt('Press a key')} arr\^ete l'ex\'ecution du
programme jusqu'\`a ce qu'on utilise de nouveau le clavier.\\

\item A la fin on revient \`a la configuration initiale.\\

\noindent {\bf Question}: Reprendre les essais pr\'ec\'edents puis augmenter
progressivement le param\`etre $\sigma$ et observer (prendre
$\sigma=1.2,1.5,1.8$ pour le sch\'ema de Lax-Wendroff). Quelles
conclusions en tirer?
\end{enumerate}
\paragraph{Quatri\`eme script Scilab: un sch\'ema implicite.} Nous allons modifier le sch\'ema de Lax-Wendroff en ``implicitant'' le terme de diffusion de la fa\c{c}on suivante:
$$
u_i^{n+1}=u_i^n-\frac{\sigma}{2}(u_{i+1}^n-u_{i-1}^n)+\frac{\sigma^2}{2}(u_{i+1}^{n+1}-2u_i^{n+1}+u_{i-1}^{n+1})
$$ 
En introduisant $W^n =
(u_i^n-\frac{\sigma}{2}(u_{i+1}^n-u_{i-1}^n))_{0\le i\le N-1}$ et
$U^{n+1}=(u_i^{n+1})_{0\le i\le N-1}$ ainsi que la matrice $A$ d'ordre
$N-1$ donn\'ee par:
$$
A = \left(\begin{array}{cccccc}
1+\sigma^2  & -\sigma^2/2 & 0           & \hdots & 0      & -\sigma^2/2 \\
-\sigma^2/2 & 1+\sigma^2  & -\sigma^2/2 & 0      & \hdots & 0 \\
0           &             &             &        &        & \vdots \\
\vdots      &             &             &        &        & 0 \\
0       & \hdots & 0 &-\sigma^2/2 & 1+\sigma^2 &-\sigma^2/2\\
-\sigma^2/2 & 0 & \hdots & 0 & -\sigma^2/2 & 1+\sigma^2
\end{array}\right)
$$ 
le sch\'ema s'\'ecrit comme $AU^{n+1}=W^n$. On notera que la
condition de p\'eriodicit\'e est utilis\'ee directement dans le sch\'ema num\'erique afin d'\'eliminer $u_i^N$. \\ 
\noindent On constate que le sch\'ema est {\it implicite}: l'\'evaluation de $U^{n+1}$ \`a partir
de $U^n$ n\'ecessite l'inversion d'un syst\`eme lin\'eaire, le co\^ut
de l'it\'eration sera donc plus \'el\'ev\'e que dans le cas d'un
sch\'ema explicite. Cependant le sch\'ema n'est pas limit\'e par la
valeur du nombre de Courant $\sigma$ ce qui permet l'utilisation des
pas de temps plus grands et donc r\'ealiser moins d'it\'erations en
temps.\\

\noindent On va cr\'eer un nouveau script appel\'e \texttt{script4.sce} \`a
partir du \texttt{script3.sce} dont on enlevera les parties relatives
au sch\'ema num\'erique.

\paragraph{Indications Scilab.} On proc\'ed\'era de la fa\c{c}on suivante
\begin{enumerate}
\item Initialisation de la matrice $A$:
\begin{verbatim}
aa = sigma^2;
dimA = N;                         // dimension de la matrice
A = diag((1+aa)*ones(1,dimA))-diag(aa/2*ones(1,dimA-1),-1)-...
    diag(aa/2*ones(1,dimA-1),+1); // on construit la matrice par ses diagonales
A(1,dimA) = -aa/2;                // on modifie la premiere ligne et la derniere
A(dimA,1) = -aa/2;                // afin de prendre en compte la periodicite       
\end{verbatim}
\item Deux options sont possibles pour la boucle en temps: r\'esoudre
le syst\`eme lin\'eaire \`a chaque pas de temps:
\begin{verbatim}
u = A\Wn;                       // a faire a chaque iteration en temps
\end{verbatim}
 ou inverser (factoriser) la matrice $A$ une fois pour toutes avant les it\'erations en temps puis utiliser son
inverse (ou factorisation):
\begin{verbatim}
B = inv(A);//ou [L,U,P]= lu(A); // inversion ou factorisation LU de A 
tps = dt:dt:Tmax;
for t = tps  
...
u =B*Wn                         // ou u = U\(L\(P*Wn));
\end{verbatim}
Il est plut\^ot conseill\'e d'utiliser la factorisation $LU$ au lieu
de l'inverse de la matrice.
\end{enumerate}

\noindent {\bf Question}: Comparer les 3 diff\'erentes options pour l'inversion
du syst\`eme lin\'eaire en calculant le temps d'execution et en
agrandissant progr\'essivement la taille du syst\`eme \`a r\'esoudre. R\'ealiser des exp\'eriences num\'eriques en
faisant varier le nombre de Courant. Discuter les performances
relatives des sch\'emas implicite et explicite.
\end{document}
