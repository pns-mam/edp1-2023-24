\documentclass[12pt,a4paper]{article}

%\usepackage[T1]{fontenc} % Pour la bonne cesure du francais
\usepackage{amsmath} % Pour les symboles complementaire comme les matrices !
\usepackage{amssymb}
\usepackage{verbatim}
\usepackage{epsfig}
%\usepackage{/home/cohen/fortran/graphics/GGGraphics/GGGraphics}
%\usepackage{D:/GGGraphics/GGGraphics}

\newtheorem{theorem}{Theorem}
\newtheorem{corollary}[theorem]{Corollary}
\newtheorem{example}{Example}
\newtheorem{rem}{\noindent\textbf{\textit {Remarque\,}}}
\newcommand{\qed}{\hfill$\qedsquare$\goodbreak\bigskip}

\def\e{{\mathchoice{\hbox{\mathbb{R}m e}}{\hbox{\mathbb{R}m e}}%
        {\hbox{\mathbb{R}m \scriptsize e}}{\hbox{\mathbb{R}m \tiny e}}}}
        
\advance\voffset by -35mm \advance\hoffset by -25mm
\setlength{\textwidth}{175mm} \setlength{\textheight}{260mm}
\pagestyle{empty}

\begin{document}

\noindent {\large Universit\'e C\^ote d'Azur} \hfill Polytech Nice Sophia (PNS)\\
\noindent Math\'ematiques Appliqu\'ees et Mod\'elisation (MAM4) \hfill jeudi 14 septembre 2023 \\

\hrule

\bigskip

\begin{center}{\bf \'Equations aux d\'eriv\'ees partielles --
TD 1}\end{center}

\bigskip

Le but de cette s\'erie d'exercices est de montrer quelques
propri\'et\'es qualitatives des solutions r\'eguli\`eres de certaines \'equations aux d\'eriv\'ees
partielles de nature diff\'erente (chaleur, \'equation des ondes, Schr\"odinger) qui font notamment intervenir la notion {\it d'\'energie}.
\begin{enumerate}
\item
On consid\`ere le probl\`eme de la chaleur en une dimension
d'espace pos\'e dans le domaine $\Omega=(0,1)$:
\begin{equation}
\left\{\begin{array}{lcl}
\displaystyle\frac{\partial u}{\partial t} -  \frac{\partial^2 u}{\partial x^2}&=
&0,\,
(x,t)\in (0,1)\times\mathbb{R}_*^+,\\
u(x,t)&=&0,\, (x,t)\in \{0,1\}\times \mathbb{R}_*^+,\\
u(x,0)&=&u_0(x),\, x\in (0,1).
\end{array}\right.
\end{equation}
{\bf a)} Montrer d'abord que toute fonction $v(x)$ contin\^ument d\'erivable sur
$[0,1]$ t.q. $v(0)=0$, v\'erifie l'in\'egalit\'e de Poincar\'e
\begin{equation}\label{eq:Poincare}
\int_0^1 v^2(x)dx \le  \int_0^1\left(\frac{d v}{d
    x}\right)^2dx.
\end{equation}
(On \'ecrira d'abord $v(x)$ comme integrale de sa d\'eriv\'ee sur
l'intervalle $(0,x)$ et ensuite on appliquera l'in\'egalit\'e de Cauchy-Schwarz).\\
{\bf b)} On notera dans ce qui suit par 
$$E(t) = \int_0^1 u^2(x,t)dx$$
l'\'energie \`a l'instant $t$. En multipliant l'\'equation de la
chaleur par $u$ et
en int\'egrant par rapport \`a $x$, \'etablir l'\'egalit\'e
v\'erifi\'ee par cette quantit\'e:
$$
\frac{1}{2}\frac{dE(t)}{dt}
=-\int_0^1\left(\frac{\partial u}{\partial x}\right)^2 dx.
$$
{\bf c)} En appliquant l'in\'egalit\'e de Poincar\'e \eqref{eq:Poincare} \`a $v(x)= u(x,t)$ en
d\'eduire que
$$
E(t) \le E(0)e^{-2t}.
$$
On dit que l'\'energie {\it d\'ecro\^it exponentiellement en temps}.
\item 
On se propose de caract\'eriser la solution r\'eguli\`ere du
probl\`eme des ondes en une dimension d'espace dans le domaine $\Omega\subset \mathbb{R}$
\begin{equation}
\left\{\begin{array}{lcl}
\displaystyle\frac{\partial^2 u}{\partial t^2} -  \frac{\partial^2 u}{\partial x^2} &= &0,\,
(x,t)\in\Omega\times\mathbb{R}_*^+,\\
u(x,t)&=&0,\,(x,t)\in\partial\Omega\times \mathbb{R}_*^+,\\
u(x,0)&=&u_0(x),\, x\in\Omega,\\
\frac{\partial u}{\partial t}(x,0)& = & u_1(x),\, x\in \Omega,
\end{array}\right.
\end{equation}
o\`u $u_0$ et $u_1$ sont des
fonctions r\'eguli\`eres et $U_1$ une primitive de $u_1$.\\
{\bf a)} Supposons maintenant que $\Omega= (0,1)$. En multipliant l'\'equation par $\frac{\partial u}{\partial t}$ et
en int\'egrant par rapport \`a $x$, \'etablir {\it l'\'egalit\'e de
l'\'energie}:
\begin{equation}
\frac{d}{dt}\left(\int_0^1 \left(\frac{\partial u}{\partial
      t}(x,t)\right)^2dx + \int_0^1 \left(\frac{\partial u}{\partial
      x}(x,t)\right)^2dx\right)=0.
\end{equation}
{\bf b)}. En utilisant l'\'egalit\'e de l'\'energie montrer que cette
solution est unique. (On choisira $u_0=u_1=0$ et on montrera que
l'unique solution est celle nulle).\\
{\bf c)} V\'erifier que la fonction suivante est solution du
probl\`eme des ondes si $\Omega=\mathbb{R}$
\begin{equation}
u(x,t) =
\frac{1}{2}(u_0(x+t)+u_0(x-t))+\frac{1}{2}(U_1(x+t)-U_1(x-t)).
\end{equation}
Cette derni\`ere relation porte le nom de {\it formule de d'Alembert}.
\item
L'\'equation de Schr\"odinger d\'ecrit l'\'evolution de la fonction
d'onde $u:\mathbb{R}^N\times \mathbb{R}^+\rightarrow \mathbb{C}$ d'une
particule soumise \`a un potentiel $V:\mathbb{R}^N\rightarrow
\mathbb{R}$. La quantit\'e $|u|^2$ s'interpr\`ete comme la densit\'e
de probabilit\'e pour d\'etecter que la particule se trouve au point $(x,t)$.  

On se propose de montrer les principes de conservation de l'\'energie
pour une solution r\'eguli\`ere de l'\'equation de Schr\"odinger uni-dimensionnelle:
\begin{equation}
\left\{\begin{array}{lcl}
\displaystyle i\frac{\partial u}{\partial t}+\frac{\partial^2 u}{\partial x^2} -V u &=&0,\, (x,t)\in\mathbb{R}\times
\mathbb{R}_*^+,\\
\displaystyle\lim_{|x|\rightarrow\infty}u(x,t)  &=& 0,\, t\in
\mathbb{R}_*^+, \\
\displaystyle\lim_{|x|\rightarrow\infty}\frac{\partial
  u(x,t)}{\partial x}  &=& 0,\, t\in
\mathbb{R}_*^+, \\
u(x,0) & = & u_0,\,x\in \mathbb{R}.
\end{array}\right.
\end{equation}
{\bf a)} Montrer que pour toute fonction d\'erivable $v$ on a
\begin{equation}
\Re\left(\frac{\partial v}{\partial t}\bar v\right) =
\frac{1}{2}\frac{\partial |v|^2}{\partial t},
\end{equation}
o\`u $\Re v$ d\'esigne la partie r\'eelle de la fonction $v$ et $\bar v$
son conjugu\'e complexe.\\
{\bf b)} En multipliant l'\'equation par $\bar u$ et en int\'egrant par
rapport \`a $x$, \'etablir l'\'egalit\'e de l'\'energie:
\begin{equation}
\int_{\mathbb{R}}|u(x,t)|^2dx = \int_{\mathbb{R}}|u_0(x)|^2dx.
\end{equation}
{\bf c)} En multipliant l'\'equation par $\frac{\partial \bar
  u}{\partial t}$, montrer que
\begin{equation}
\int_{\mathbb{R}}\left(\left|\frac{\partial u}{\partial
      x}(x)\right|^2 +V(x) |u(x,t)|^2\right)dx = \int_{\mathbb{R}}\left(\left|\frac{\partial u_0}{\partial
      x}(x)\right|^2 +V(x) |u_0(x)|^2\right)dx.
\end{equation}
\end{enumerate}
%\pagebreak
%\bigskip
%\hrule
%\noindent{\bf Evaluation du cours \'Equations aux D\'eriv\'ees Partielles :}
%\begin{itemize}
%\item[$\bullet$] Un contr\^ole \'ecrit le vendredi $23$ Octobre (pendant la s\'eance de cours). 
%\item[$\bullet$] Une note de devoir maison/projet (présentation orale le 4 décembre).
%\item[$\bullet$] Un examen \'ecrit pendant la session d'examen. 
%\end{itemize}
%La note finale est : $30\%$(note contr\^ole) $+$
%$30\%$(note devoir/projet) $+$
%$40\%$(note examen).

\end{document}
