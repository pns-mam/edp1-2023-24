\documentclass[12pt,a4paper]{article}

%\usepackage[T1]{fontenc} % Pour la bonne cesure du francais
\usepackage{amsmath} % Pour les symboles complementaire comme les matrices !
\usepackage{amssymb}
\usepackage{verbatim}
\usepackage{epsfig}
%\usepackage{/home/cohen/fortran/graphics/GGGraphics/GGGraphics}
%\usepackage{D:/GGGraphics/GGGraphics}

\newtheorem{theorem}{Theorem}
\newtheorem{corollary}[theorem]{Corollary}
\newtheorem{example}{Example}
\newtheorem{rem}{\noindent\textbf{\textit {Remarque\,}}}
\newcommand{\qed}{\hfill$\qedsquare$\goodbreak\bigskip}

\def\e{{\mathchoice{\hbox{\mathbb{R}m e}}{\hbox{\mathbb{R}m e}}%
        {\hbox{\mathbb{R}m \scriptsize e}}{\hbox{\mathbb{R}m \tiny e}}}}
        
\advance\voffset by -35mm \advance\hoffset by -25mm
\setlength{\textwidth}{175mm} \setlength{\textheight}{260mm}
\pagestyle{empty}

\begin{document}

\noindent {\large Universit\'e C\^ote d'Azur} \hfill Polytech Nice Sophia (PNS)\\
\noindent Math\'ematiques Appliqu\'ees et Mod\'elisation (MAM4) \hfill vendredi 29 Septembre 2023 \\

\hrule

\bigskip
\bigskip

\begin{center}{\bf \'Equations aux d\'eriv\'ees partielles --
TD 3}\end{center}

\bigskip

\parskip 12pt
\noindent Consid\'erons l'\'equation de la chaleur en une dimension d'espace dans le domaine born\'e $(0,1)$:
$$
\begin{cases}
\displaystyle\frac{\partial u}{\partial t}-\nu \frac{\partial^2 u}{\partial x^2}=0,\, \forall (x,t)\in(0,1)\times\mathbb{R}^+_*,\\[2ex]
u(x,0)=u_0(x),\,\forall x\in (0,1).
\end{cases}
$$
On discr\'etise le domaine en utilisant un maillage r\'egulier
$(t_n,x_j)=(n\Delta t,j\Delta x)$,  $\forall n\ge 0,
j\in\{0,1,...,N+1\}$ o\`u $\Delta x=1/(N+1)$ et $\Delta t>0$. 

\noindent Les conditions aux limites sont des conditions de Dirichlet homogènes:
$u(0,t)=u(1,t)=0,\,\forall t$. \\

\noindent Le but de cette s\'erie d'exercices est d'illuster dans un premier temps la propri\'et\'e de consistance
d'\'evaluer la pr\'ecision de quelques sch\'emas. Ceci se fera en calculant l'erreur de troncature \`a l'aide de d\'eveloppements de
Taylor. 
 
\noindent Ensuite on va travailler sur la
propri\'et\'e de stabilit\'e en norme $L^2$. On va évaluer le facteur
d'amplification de chaque sch\'ema en y injectant un mode de Fourier
et on \'etablira sous quelles conditions ce facteur d'amplification
est inferieur \`a un en module. 


\noindent On se propose d'\'etudier les sch\'emas suivants:

\begin{enumerate}
%\item {\it Sch\'ema d'Euler explicite}
%$$
%\frac{u_j^{n+1}-u_j^n}{\Delta t}-\nu \frac{u_{j+1}^{n}-2u_j^n+u_{j-1}^{n}}{\Delta x^2}=0.
%$$
%
%\item {\it Sch\'ema d'Euler implicite}
%$$
%\frac{u_j^{n+1}-u_j^n}{\Delta t}-\nu \frac{u_{j+1}^{n+1}-2u_j^{n+1}+u_{j-1}^{n+1}}{\Delta x^2}=0.
%$$

\item {\it Sch\'ema de Crank-Nicolson}
$$
\frac{u_j^{n+1}-u_j^n}{\Delta t}-\nu \frac{u_{j+1}^{n}-2u_j^n+u_{j-1}^{n}}{2\Delta x^2}-\nu \frac{u_{j+1}^{n+1}-2u_j^{n+1}+u_{j-1}^{n+1}}{2\Delta x^2}=0.
$$

\item {\it Le $\theta$ sch\'ema}
$$
\frac{u_j^{n+1}-u_j^n}{\Delta t}-\theta\nu \frac{u_{j+1}^{n}-2u_j^n+u_{j-1}^{n}}{\Delta x^2}-(1-\theta)\nu \frac{u_{j+1}^{n+1}-2u_j^{n+1}+u_{j-1}^{n+1}}{\Delta x^2}=0.
$$

On va remarquer que les développement faits pour ce schéma permettront de déduire les conclusions pour les schémas d'Euler implicite et explicite (fait en cours) et Crank-Nicolson qui sont des cas particuliers de $\theta$ schéma.
%\item {\it Sch\'ema aux six points}
%$$
%\begin{array}{lcl}
%\displaystyle\frac{u_{j+1}^{n+1}-u_{j+1}^n}{12\Delta t}& +& \displaystyle\frac{5(u_{j}^{n+1}-u_{j}^n)}{6\Delta t}+\frac{u_{j-1}^{n+1}-u_{j-1}^n}{12\Delta t}+\\[2ex]
%& -&\displaystyle\nu \frac{u_{j+1}^{n}-2u_j^n+u_{j-1}^{n}}{2\Delta x^2}-\nu \frac{u_{j+1}^{n+1}-2u_j^{n+1}+u_{j-1}^{n+1}}{2\Delta x^2}=0.
%\end{array}
%$$

\item {\it Sch\'ema de DuFort-Frankel}
$$
\frac{u_j^{n+1}-u_j^{n-1}}{2\Delta t}-\nu \frac{u_{j+1}^{n}-u_j^{n+1}-u_j^{n-1}+u_{j-1}^{n}}{\Delta x^2}=0.
$$

%\item {\it Sch\'ema de Gear}
%$$
%\frac{3u_j^{n+1}-4u_j^n+u_j^{n-1}}{2\Delta t}-\nu \frac{u_{j+1}^{n+1}-2u_j^{n+1}+u_{j-1}^{n+1}}{\Delta x^2}=0.
%$$
\newpage
\noindent {\bf Consistance et ordre d'approximation}\\
Montrer que:
\begin{itemize}
\item  le $\theta$ sch\'ema est pr\'ecis à l'ordre
$1$ en temps et l'ordre $2$ en espace.
\item le sch\'ema de Crank-Nicolson
est pr\'ecis à l'ordre $2$ en temps et l'ordre $2$
en espace
%\item le sch\'emas à six points est pr\'ecis à l'ordre $2$ en temps et l'ordre $4$ en espace. 
\item Que peut-on dire du sch\'ema de
DuFort- Frankel?
\end{itemize}

Quelques r\`egles \`a suivre 
\begin{itemize}
\item Il faudra d\'evelopper tous les termes du sch\'ema {\it au
    m\^eme point}; le choix de ce dernier n'a pas d'importance et n'a
  aucune influence sur le r\'esultat final, mais peut influencer la
  longueur et la difficult\'e du calcul.
\item Il est recommand\'e de diviser le calcul en plusieurs \'etapes,
  les r\'esultats intermediaires pouvant \^etre utils par la suite.
\item Il faut {\it absolument} utiliser \`a un certain moment
  l'\'equation v\'erifi\'ee par la solution, qui simplifiera les
  calculs et permettera de trouver le r\'esultat optimal.
\item Il faut \'eviter de manipuler les termes non-significatifs, ceci
  conduisant \`a des calculs inutiles.
\end{itemize}

{\bf Stabilit\'e en norme $L^2$}\\
%1) Montrer que le sch\'ema explicite est stable en norme $L^2$ sous la condition $2\nu\Delta t\le \Delta x^2$ et que le sch\'ema implicite est inconditionnellement stable.\\
Montrer que:
\begin{itemize}
\item le $\theta$- sch\'ema est
{\it inconditionnellement stable} en norme $L^2$ si $\theta \le 1/2$ et que si $1/2 < \theta \le 1$ alors il est
stable sous la condition $$2(2\theta-1)\frac{\nu\Delta
    t}{\Delta x^2} \le 1.$$ 
    \item le sch\'ema de DuFort-Frankel est inconditionnellement
stable en norme $L^2$.
\item si on fait tendre $\Delta t$ et
$\Delta x$ simultanement vers $0$ de mani\`ere que le rapport $\Delta
t/\Delta x$ tende aussi vers $0$, alors le sch\'ema de DuFort-Frankel
est convergent (on dit qu'il est "conditionnellement'' convergent).  
\end{itemize}



%\noindent \texttt{Indication}: On pourra \'etudier directement le $\theta$-sch\'ema en pr\'ecisant
%ensuite pour quelles valeurs de $\theta$ on obtient les sch\'emas
%explicites, implicites et Crank-Nicolson. On tiendra compte des
%changements pour ce dernier o\`u l'ordre d'approximation n'est pas le m\^eme.
\end{enumerate}

%\bigskip
%\hrule
%\noindent{\bf Evaluation du cours \'Equations aux D\'eriv\'ees Partielles :}
%\begin{itemize}
%\item[$\bullet$] Un contr\^ole \'ecrit le vendredi $23$ Octobre (pendant la s\'eance de cours). 
%\item[$\bullet$] Une note de devoir maison/projet (présentation orale le 4 décembre).
%\item[$\bullet$] Un examen \'ecrit pendant la session d'examen. 
%\end{itemize}
%La note finale est : $30\%$(note contr\^ole) $+$
%$30\%$(note devoir/projet) $+$
%$40\%$(note examen).

\end{document}
