\documentclass[leqno,11pt]{article}
\usepackage{amssymb}
\parindent 0cm
\newcommand{\cD}{\cal D}
\newcommand{\cT}{\cal T}
\newtheorem{Th}{Theorem}
\newtheorem{Cor}{Corollary}
\newtheorem{lemma}{Lemma}
\newtheorem{proposition}{Proposition}
\newtheorem{rem}{Remark}
%--------dimension
%\textwidth=15cm
%\textheight=22cm
%\oddsidemargin=20mm     \evensidemargin=20mm
%\hoffset=-25mm            \voffset=-25mm
%\pretolerance=500      \tolerance=1000  \brokenpenalty=5000



\newcommand{\cqfd}
{%
\mbox{}%
\nolinebreak%
\hfill%
\rule{2mm}{2mm}%
\medbreak%
\par%
}
\newcommand{\DD}{{\mathbb D}}
\newcommand{\CC}{{\mathbb C}}
\newcommand{\RR}{{\mathbb R}}
\newcommand{\TT}{{\mathbb T}}
\newcommand{\NN}{{\mathbb N}}
\newcommand{\ds}{\displaystyle}
%----------------------------Dimension
\textwidth=17truecm
\textheight=25truecm
\oddsidemargin=20mm     \evensidemargin=20mm
\hoffset=-25mm            \voffset=-20mm
\pretolerance=500      \tolerance=1000  \brokenpenalty=5000
\parindent=0pt
%--------------------


\begin{document}
\thispagestyle{empty}
%-----------------------------Debut titre
\noindent\rule{17truecm}{0.4mm}
\smallbreak
\noindent{\large\bf Polytech'Nice Sophia Antipolis} \hfill {\large \bf EPU - MAM4/SI4}
\smallbreak

\centerline {\large \sc Equations aux D\'eriv\'ees Partielles}

\bigbreak

\noindent {\bf Autour de la formule et du noyau de Green} \hfill {\bf Feuille n$^o$ 2\ \ Sept. 2010}

\smallbreak


\noindent\rule{17truecm}{0.4mm} \\


%-----------------------------Fin titre
\vspace{1cm}
{\sl Dans tous les exercices de cette feuille, on supposera que les fonctions sont suffisamment d\'erivables.}
\vspace{1cm}

\noindent {\bf Exercice 1 } : \\

On consid\`ere une fonction $f$ num\'erique d\'efinie sur $\ds \RR^n, \, n\ge1$ et $\ds v \, = \, v(x) \, = \, (v_1(x), \,v_2(x), \cdots , \, v_n(x))$ un champ de vecteurs d\'efini sur $\ds \RR^n$. On rappelle la d\'efinition des op\'erateurs suivants :
\begin{description}
\item [Le gradient] not\'e $\nabla$ ou grad, op\`ere sur les fonctions num\'eriques de $n$ variables r\'eelles ($\ds f \, :  \, \RR^n \longmapsto \RR$), et produit des champs de vecteurs d\'efinis sur $\ds \RR^n$ ($\ds \nabla f \, : \, \RR^n \longmapsto \RR^n$) :
$$
\nabla f \, = \, \mbox{grad} f \, = \,
\left(
\begin{array}{c}
\ds \frac{\partial f}{\partial x_1} \\
\ds \frac{ \partial f}{\partial x_2}\\
\ds \vdots \\
\ds \frac{ \partial f}{\partial x_2}
\end{array}
\right)
$$
\item [Le Laplacien] not\'e $\Delta$, op\`ere sur les fonctions num\'eriques de $n$ variables r\'eelles ($\ds f \, : \, \RR^n \longmapsto \RR$), et produit des champs scalaires d\'efinis sur $\ds \RR^n$ ($\ds \Delta f \, : \, \RR^n \longmapsto \RR$) :
$$
\Delta f \, = \, \frac{\partial^2 f}{\partial x_1^2} \, + \, \frac{\partial^2 f}{\partial x_2^2} \, + \cdots + \,\frac{\partial^2 f}{\partial x_n^2}
$$

\item [La divergence] not\'ee div ou $\nabla \cdot$, op\`ere sur les champs de vecteurs d\'efinis sur $\ds \RR^n$ ($\ds v \, : \, \RR^n \longmapsto \RR^n$), et produit des champs scalaires d\'efinis sur $\ds \RR^n$ ($\ds \nabla \cdot  v \, : \, \RR^n \longmapsto \RR$):
$$
\mbox{div} \, v \, = \, \nabla \cdot v \, = \, \frac{\partial v_1}{\partial x_1} \, + \,  \frac{\partial v_2}{\partial x_2} \, + \cdots + \, \frac{\partial v_n}{\partial x_n}
$$
\end{description}

\begin{enumerate}
\item V\'erifier la relation $\ds \mbox{div} \, \nabla f \, = \, \Delta f$ ;
\item En utilisant la formule de la d\'eriv\'ee du produit de fonctions num\'eriques d'une variable r\'eelle $uv$, rappeler la formule d'int\'egration par parties sur un intervalle $[a,b]$ de $\RR$ ;
\item  $f$ et $g$ \'etant deux fonctions num\'eriques de $n$ variables r\'eelles, montrer que :
$$
\mbox{div}\, (g \cdot \nabla f) \, = \, g \, \Delta f \, + \, \nabla g \cdot \nabla f
$$
\item Soit $\Omega$ un domaine ouvert de $\ds \RR^n$ de fronti\`ere $\ds \partial \Omega \, = \, \Gamma$ r\'eguli\`ere. On rappelle la formule de Green, qui est l'analogue de l'int\'egration par parties :
$$
\int_\Omega \frac{\partial u}{\partial x_i} \, v \, = \, - \, \int_\Omega u \, \frac{\partial v}{\partial x_i} \, + \,\int_{\partial \Omega} u \,v \, n_i
$$
$\ds n_i$ \'etant la $i$-\`eme composante de la normale sortante \`a $\partial \Omega$.
\begin{enumerate}
\item Montrer que si $v$ est un champ vectoriel d\'efini sur $\ds \RR^n$ ($\, v \, : \, \RR^n \, \longmapsto \, \RR^n$), alors on a la {\sl formule de la divergence} suivante :
$$
\int_\Omega \mbox{div}  \, v \, =  \,\int_{\partial \Omega} v \cdot n \, = \,\,\int_{\partial \Omega} v_i \, n_i
$$
\item Montrer que si $f$ et $g$ sont des champs scalaires d\'efinis sur $\ds \RR^n$, alors on a :
$$
\int_\Omega \Delta f   \, g \, = \, - \, \int_\Omega \nabla f  \, \cdot \, \nabla g \,+ \, \int_{\partial \Omega} \frac{\partial f}{\partial n}  \, g
$$

o\`u la {\sl d\'eriv\'ee normale} $\ds \frac{\partial f}{\partial n} $ est d\'efinie de la mani\`ere suivante :
$$
\frac{\partial f}{\partial n} \, = \, \sum_{i=1}^n \frac{\partial f}{\partial x_i} \, n_i
$$
\end{enumerate}
\end{enumerate}


\vspace{0,5cm}

\noindent {\bf Exercice 2 : Solution \'el\'ementaire de l'\'equation de la chaleur} : \\

On consid\`ere la fonction $G$ d\'efinie par :
$$
\begin{array}{rrcl}
\ds G : & \RR \, \times \, ]0 \, , \, +\infty[ & \longmapsto & \RR \\
\ds & (x \, , \, t) & \longmapsto & \ds G(x,t) \, = \, \frac{1}{\sqrt{4 \pi t}}\, \exp (-\frac{x^2}{4t})
\end{array}
$$
\begin{enumerate}
\item Montrer que $G(x,t)$ est solution de l'\'equation de la chaleur dans $\RR$, c'est \`a dire que :
$$
\frac{\partial G}{\partial t} \, - \, \frac{\partial^2 G}{\partial x^2} \, = \, 0 \qquad \forall \, (x \, , \, t ) \, \in \, \RR \, \times \, ]0 \, , \, + \infty[
$$
\item Tracer (avec une  pr\'ecision mod\'er\'ee) les courbes repr\'esentatives de la fonction $\ds x \, \, \longmapsto \,G(x,t)$ pour $\ds t = \frac{1}{4}$ et pour $t=1$. Comparer les r\'esultats obtenus. Que peut-on dire du comportement de la fonction  $\ds x  \, \longmapsto \,G(x,t)$ lorsque $t \to 0$ ?
\item Calculer la valeur de l'int\'egrale $\ds \int_{-\infty}^{+\infty} G(x,t) \, dx$ (on rappelle que  $\ds \int_{-\infty}^{+\infty} \exp(-y^2) \, dy \, = \, \sqrt{\pi} $).
\item On d\'efinit \`a pr\'esent la fonction :
$$
u(x,t) \, = \, \int_{-\infty}^{+\infty} G(x-y,t) \, u_0(y) \, dy
$$
Montrer, en admettant que toutes les d\'erivations sour le signe $\ds \int$ soient licites, que $u$ est solution du probl\`eme de Cauchy pour l'\'equation de la chaleur :
$$
\left\{
\begin{array}{rcll}
\ds \frac{\partial u}{\partial t} \, - \, \frac{\partial^2 u}{\partial x^2} & = & 0 &  \forall \, (x \, , \, t ) \, \in \, \RR \, \times \, ]0 \, , \, + \infty[ \\
\ds u(x,0)&=&\ds u_0(x)&
\end{array}
\right.
$$
\end{enumerate}
\end{document}
\vspace{0,5cm}
\noindent {\bf Exercice 3 : } : \\



