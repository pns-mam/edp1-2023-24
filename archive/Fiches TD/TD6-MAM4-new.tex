\documentclass[12pt,a4paper]{article}

%\usepackage[T1]{fontenc} % Pour la bonne cesure du francais
\usepackage{amsmath} % Pour les symboles complementaire comme les matrices !
\usepackage{amssymb}
\usepackage{verbatim}
\usepackage{epsfig}
%\usepackage{/home/cohen/fortran/graphics/GGGraphics/GGGraphics}
%\usepackage{D:/GGGraphics/GGGraphics}

\newtheorem{theorem}{Theorem}
\newtheorem{corollary}[theorem]{Corollary}
\newtheorem{example}{Example}
\newtheorem{rem}{\noindent\textbf{\textit {Remarque\,}}}
\newcommand{\qed}{\hfill$\qedsquare$\goodbreak\bigskip}

\def\e{{\mathchoice{\hbox{\mathbb{R}m e}}{\hbox{\mathbb{R}m e}}%
        {\hbox{\mathbb{R}m \scriptsize e}}{\hbox{\mathbb{R}m \tiny e}}}}
        
\advance\voffset by -35mm \advance\hoffset by -25mm
\setlength{\textwidth}{175mm} \setlength{\textheight}{260mm}
\pagestyle{empty}

\begin{document}

\noindent {\large Universit\'e Côte d'Azur} \hfill Polytech Nice Sophia (PNS)\\
\noindent Math\'ematiques Appliqu\'ees et Mod\'elisation (MAM4)
\hfill mardi 18 Octobre 2022\\

\hrule

\bigskip
\bigskip

\begin{center}{\bf \'Equations aux d\'eriv\'ees partielles --
S\'erie 6}\end{center}

\bigskip

\parskip 12pt
\noindent Consid\'erons l'\'equation des ondes dans le domaine born\'e $(0,1)$:
$$
\begin{cases}
\displaystyle\frac{\partial^2 u}{\partial t^2}-\frac{\partial^2 u}{\partial
  x^2}=0,\, \forall (x,t)\in(0,1)\times\mathbb{R}^+_*,
\end{cases}
$$
avec $u(x, 0) = u_0,\, \displaystyle\frac{\partial u}{\partial t}(x,0)=u_1(x)$,
$u$,$u_0$ et $u_1$ p\'eriodiques de p\'eriode 1 et $u_1$ \`a moyenne nulle
$$
\int_0^1 u_1(x)\,dx =0.
$$
On discr\'etise le domaine en utilisant un maillage r\'egulier
$(t_n,x_j)=(n\Delta t,j\Delta x)$,  $\forall n\ge 0,
j\in\{0,1,...,N+1\}$ o\`u $\Delta x=1/(N+1)$ et $\Delta t>0$.\\

\noindent {\bf Probl\`eme 1}. Consid\'erons d'abord le $\theta$-sch\'ema centr\'e
\begin{equation}
\frac{u_j^{n+1}-2u_j^n+u_j^{n-1}}{\Delta t^2}-\theta
\frac{u_{j+1}^{n+1}-2u_j^{n+1}+u_{j-1}^{n+1}}{\Delta
  x^2}-(1-2\theta)\frac{u_{j+1}^{n}-2u_j^{n}+u_{j-1}^{n}}{\Delta
  x^2}-\theta \frac{u_{j+1}^{n-1}-2u_j^{n-1}+u_{j-1}^{n-1}}{\Delta
  x^2}=0.
\end{equation}
avec $0\le \theta \le 1/2$.
\begin{enumerate}

\item Montrer que si $1/4\le \theta \le 1$, le $\theta$-sch\'ema
  centr\'e est inconditionnellement stable en norme $L^2$. Si $0\le
  \theta < 1/4$, il est stable sous la condition
$$
\frac{\Delta t}{\Delta x} < \sqrt{\frac{1}{1-4\theta}}.
$$
Consid\'erons maintenant le cas limite o\`u $\Delta t/\Delta
  x=(1-4\theta)^{-1/2}$ avec $0\le \theta < 1/4$. Montrer que le
  sch\'ema est instable dans ce cas, en v\'erifiant que $u_j^n =
  (-1)^{j+n}(2n-1)$ est une solution.
\item On admettera que l'\'energie discrete suivante
$$
E^n=\sum_{j=0}^N\left(\frac{u_j^{n+1}-u_j^n}{\Delta
    t}\right)^2+a_{\Delta x}(u^{n+1},u^n)+\theta a_{\Delta x}(u^{n+1}-u^n,u^{n+1}-u^n)
$$
est une approximation d'ordre $1$ en espace et en temps de l'\'energie
continue d\'efinie en cours. Ici on a not\'e pour tout
$u,v\in\mathbb{R}^{n+1}$, $u=(u_j)_{0\le j\le N},\, v=(v_j)_{0\le j\le N}$
$$
a_{\Delta x}(u,v)=\sum_{j=0}^N \frac{u_{j+1}-u_j}{\Delta x}\cdot
\frac{v_{j+1}-v_j}{\Delta x}.
$$
Montrer que le $\theta$-sch\'ema centr\'e conserve l'\'energie
discrete, c.a.d. $E^n=E^0$ pour tout $n>0$.
\end{enumerate}
\noindent {\bf Probl\`eme 2}. Consid\'erons le sch\'ema de
Lax-Friedrichs appliqu\'e \`a l'\'equation des ondes \'ecrite comme
syst\`eme du premier ordre
\begin{equation}
\displaystyle \frac{1}{2\Delta t}\left(\begin{array}{c}
    2v_j^{n+1}-v_{j+1}^n-v_{j-1}^n \\2w_j^{n+1}-w_{j+1}^n-w_{j-1}^n \end{array}\right)-\frac{1}{2\Delta x}\left(\begin{array}{cc}
    0 & 1 \\1 & 0 \end{array}\right)
\left(\begin{array}{c} v_{j+1}^n - v_{j-1}^n \\ w_{j+1}^n - w_{j-1}^n\end{array}\right) = 0
\end{equation}
Ici $v$ a la signification d'un d\'eplacement et $w$, d'une
d\'eformation. Montrer que ce sch\'ema est stable en norme $L^2$ sous la condition CFL $\Delta
t\le \Delta x$ et qu'il est pr\'ecis \`a l'ordre $1$ en espace et en
temps si le rapport $\Delta t/\Delta x$ est gard\'e constant lorsque
$\Delta t$ et $\Delta x$ tendent vers $0$.\\

\noindent {\bf Probl\`eme 3}.  Consid\'erons le sch\'ema de
Lax-Wendroff appliqu\'e \`a l'\'equation des ondes \'ecrite comme
syst\`eme du premier ordre
\begin{equation}
\displaystyle \frac{1}{\Delta t}\left(\begin{array}{c}
    v_j^{n+1}-v_{j}^n \\w_j^{n+1}-w_{j}^n \end{array}\right)-\frac{1}{2\Delta x}\left(\begin{array}{cc}
    0 & 1 \\1 & 0 \end{array}\right)
\left(\begin{array}{c} v_{j+1}^n - v_{j-1}^n \\ w_{j+1}^n -
    w_{j-1}^n\end{array}\right)-\frac{\Delta t}{2\Delta x^2} \left(\begin{array}{cc}
    0 & 1 \\1 & 0 \end{array}\right)^2 \left(\begin{array}{c} v_{j+1}^n -2v_j^n+ v_{j-1}^n \\ w_{j+1}^n -2w_j^n+ w_{j-1}^n\end{array}\right)= 0
\end{equation} 
Montrer que ce sch\'ema est stable en norme $L^2$ sous la condition CFL $\Delta
t\le \Delta x$ et qu'il est pr\'ecis \`a l'ordre $2$ en espace et en
temps.

\end{document}
