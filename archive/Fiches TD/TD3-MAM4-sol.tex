\documentclass[12pt,a4paper]{article}

%\usepackage[T1]{fontenc} % Pour la bonne cesure du francais
\usepackage{amsmath} % Pour les symboles complementaire comme les matrices !
\usepackage{amssymb}
\usepackage{verbatim}
\usepackage{epsfig}
%\usepackage{/home/cohen/fortran/graphics/GGGraphics/GGGraphics}
%\usepackage{D:/GGGraphics/GGGraphics}

\newtheorem{theorem}{Theorem}
\newtheorem{corollary}[theorem]{Corollary}
\newtheorem{example}{Example}
\newtheorem{rem}{\noindent\textbf{\textit {Remarque\,}}}
\newcommand{\qed}{\hfill$\qedsquare$\goodbreak\bigskip}

\def\e{{\mathchoice{\hbox{\mathbb{R}m e}}{\hbox{\mathbb{R}m e}}%
        {\hbox{\mathbb{R}m \scriptsize e}}{\hbox{\mathbb{R}m \tiny e}}}}
        
\advance\voffset by -35mm \advance\hoffset by -25mm
\setlength{\textwidth}{175mm} \setlength{\textheight}{260mm}
\pagestyle{empty}

\begin{document}

\noindent {\large Universit\'e C\^ote d'Azur} \hfill Polytech' Nice Sophia (PNS)\\
\noindent Math\'ematiques Appliqu\'ees et Mod\'elisation (MAM4) \hfill lundi 20 Septembre 2021 \\

\hrule

\bigskip
\bigskip

\begin{center}{\bf \'Equations aux d\'eriv\'ees partielles --
TD 3\\ SOLUTIONS}\end{center}

\bigskip


\noindent Par la suite on va utiliser les relations suivantes
bas\'ees sur des d\'eveloppements en s\'erie de Taylor et on va tenir compte du fait que $\frac{\partial u}{\partial t} = \nu
\frac{\partial^2 u}{\partial x^2}$ (car $u$ est solution de l'\'equation)
\begin{equation}\label{eq:util}
\begin{array}{l}
\displaystyle \frac{u(x_j,t_{n+1})-u(x_j,t_n)}{\Delta t}  = 
\displaystyle\frac{\partial u}{\partial t}(x_j,t_n) + \frac{\Delta t}{2}
\frac{\partial^2 u}{\partial t^2}(x_j,t_n)+{\cal O}(\Delta t^2),\\[2ex]
\displaystyle \frac{u(x_{j+1},t_n)-2u(x_j,t_n)+u(x_{j-1},t_n)}{\Delta
  x^2}  = \displaystyle\frac{\partial^2 u}{\partial
  x^2}(x_j,t_n)+\frac{\Delta x^2}{12}\frac{\partial^4 u}{\partial
  x^4}(x_j,t_n)+{\cal O}(\Delta x^4),\\[2ex]
\displaystyle \frac{u(x_{j+1},t_{n+1})-2u(x_j,t_{n+1})+u(x_{j-1},t_{n+1})}{\Delta
  x^2}  = \displaystyle\frac{\partial^2 u}{\partial
  x^2}(x_j,t_{n+1})+\frac{\Delta x^2}{12}\frac{\partial^4 u}{\partial
  x^4}(x_j,t_{n+1})+{\cal O}(\Delta x^4) \\[2ex]
\qquad = \displaystyle\frac{\partial^2 u}{\partial
  x^2}(x_j,t_n)+\frac{\Delta x^2}{12}\frac{\partial^4 u}{\partial
  x^4}(x_j,t_n)+\displaystyle \Delta t\left(\frac{\partial^3u}{\partial t\partial
    x^2}(x_j,t_n)+\frac{\Delta x^2}{12}\frac{\partial^5u}{\partial t\partial
    x^4}(x_j,t_n)\right)+{\cal O}(\Delta t^2) +{\cal O}(\Delta x^4)\\
\qquad = \displaystyle\frac{\partial^2 u}{\partial
  x^2}(x_j,t_n)+\left(\nu\Delta t+ \frac{\Delta x^2}{12}\right) \frac{\partial^4 u}{\partial
  x^4}(x_j,t_n)+\frac{\nu\Delta t \Delta x^2}{12}\frac{\partial^6 u}{\partial
  x^6}(x_j,t_n) +{\cal O}(\Delta t^2) +{\cal O}(\Delta x^4).
\end{array}
\end{equation}
\begin{enumerate}
\item On regardera d'abord la consistance du
  $\theta$-sch\'ema et on obtiendra comme cas particulier le sch\'ema de Crank-Nicolson. On \'evalue dans ce cas l'erreur de troncature
\begin{equation}\label{eq:theta}
\begin{array}{l}
{\cal E}^n_j=\displaystyle \frac{u(x_j,t_{n+1})-u(x_j,t_n)}{\Delta t} -\displaystyle \nu\theta
\frac{u(x_{j+1},t_n)-2u(x_j,t_n)+u(x_{j-1},t_n)}{\Delta  x^2}\\
\quad-\displaystyle \nu (1-\theta) \frac{u(x_{j+1},t_{n+1})-2u(x_j,t_{n+1})+u(x_{j-1},t_{n+1})}{\Delta
  x^2}\\
\quad = \displaystyle\frac{\partial u}{\partial t}(x_j,t_n) + \frac{\Delta t}{2}
\frac{\partial^2 u}{\partial t^2}(x_j,t_n)+{\cal O}(\Delta t^2)-\displaystyle\nu\theta \left( \frac{\partial^2 u}{\partial
    x^2}(x_j,t_n)+\frac{\Delta x^2}{12}\frac{\partial^4 u}{\partial
    x^4}(x_j,t_n)+{\cal O}(\Delta x^4)\right)\\
\quad -\displaystyle\nu(1-\theta) \left( \frac{\partial^2 u}{\partial
  x^2}(x_j,t_n)+\left(\nu\Delta t+ \frac{\Delta x^2}{12}\right) \frac{\partial^4 u}{\partial
  x^4}(x_j,t_n)+\frac{\nu\Delta t \Delta x^2}{12}\frac{\partial^6 u}{\partial
  x^6}(x_j,t_n) +{\cal O}(\Delta t^2) +{\cal O}(\Delta x^4)\right).
\end{array}
\end{equation}
On tiendra de nouveau compte du fait que $\frac{\partial u}{\partial t} = \nu
\frac{\partial^2 u}{\partial x^2}$ et par cons\'equent que $\frac{\partial^2 u}{\partial t^2} = \nu^2
\frac{\partial^4 u}{\partial x^4}$. En groupant les termes du m\^eme
ordre de \eqref{eq:theta} on obtient
$$
\begin{array}{l}
{\cal E}^n_j =\displaystyle (1-\theta -(1-\theta))\nu\frac{\partial^2
  u}{\partial x^2}(x_j,t_n)\\[2ex]
\quad +\displaystyle\left(\frac{\nu\Delta t}{2} -\theta \frac{\Delta x^2}{12}
  -(1-\theta)\left(\frac{\Delta x^2}{12}+\nu\Delta t\right)\right) \nu\frac{\partial^4
  u}{\partial x^4}(x_j,t_n)+{\cal O}(\Delta t^2) +{\cal O}(\Delta
x^4)\\[2ex]
\quad =\displaystyle\left(\theta-\frac{1}{2}\right) \Delta t\,\nu\frac{\partial^4
  u}{\partial x^4}(x_j,t_n)-\frac{\Delta x^2}{12}\,\nu\frac{\partial^4
  u}{\partial x^4}(x_j,t_n) \\
\quad \displaystyle+(\theta-1) \frac{\nu\Delta t \Delta x^2}{12}\frac{\partial^6 u}{\partial
  x^6}(x_j,t_n)+ {\cal O}(\Delta t^2) +{\cal O}(\Delta
x^4).
\end{array}
$$
On voit bien que pour $\theta=0$ (implicite) ou $\theta=1$
(explicite) le sch\'ema est d'ordre $1$ en temps et $2$ espace et il
devient d'ordre $2$ en temps pour $\theta=1/2$ (Crank-Nicolson).

%\item {\it Sch\'ema aux six points}. On voit bien que tous les termes
%  de l'erreur de troncature du sch\'ema ont d\'ej\`a \'et\'es
%  trait\'ees pr\'ec\'edamment et l'erreur de troncature de ce sch\'ema
%  peut s'obtenir directement de celui de Crank-Nicolson
%\begin{equation}\label{eq:crank}
%\tilde {\cal E}^n_j =-\frac{\Delta x^2}{12}\,\nu\frac{\partial^4
%  u}{\partial x^4}(x_j,t_n) -\frac{\nu\Delta t \Delta x^2}{24}\frac{\partial^6 u}{\partial
%  x^6}(x_j,t_n)+ {\cal O}(\Delta t^2) +{\cal O}(\Delta
%x^4).
%\end{equation}
%en y ajoutant $1/12 \bar {\cal E}^n_j$ o\`u:
%\begin{equation}\label{eq:reste}
%\begin{array}{l}
%\bar {\cal E}^n_j =  \displaystyle
%\frac{u(x_{j+1},t_{n+1})-u(x_{j+1},t_n)}{\Delta t} +\displaystyle
%  \frac{u(x_{j-1},t_{n+1})-u(x_{j-1},t_n)}{\Delta t} -2\frac{u(x_{j},t_{n+1})-u(x_{j},t_n)}{\Delta t}\\[2ex]
%\quad =\displaystyle\frac{\partial u}{\partial t}(x_{j+1},t_n) +\frac{\Delta t}{2}
%\frac{\partial^2 u}{\partial t^2}(x_{j+1},t_n)+\frac{\partial u}{\partial t}(x_{j-1},t_n) + \frac{\Delta t}{2}
%\frac{\partial^2 u}{\partial t^2}(x_{j-1},t_n)\\[2ex]
%\quad \displaystyle-2\frac{\partial u}{\partial t}(x_{j},t_n)-2\frac{\Delta t}{2}
%\frac{\partial^2 u}{\partial t^2}(x_{j},t_n)+{\cal O}(\Delta
%t^2)=\displaystyle\nu\left(\frac{\partial^2 u}{\partial
%    x^2}(x_{j+1},t_n)+\frac{\partial^2 u}{\partial
%    x^2}(x_{j-1},t_n) -2\frac{\partial^2 u}{\partial
%    x^2}(x_{j},t_n)\right)\\[2ex]
%\quad+\displaystyle\nu^2 \frac{\Delta t}{2}\left(\frac{\partial^4 u}{\partial
%    x^4}(x_{j+1},t_n)+\frac{\partial^4 u}{\partial
%    x^4}(x_{j-1},t_n)-2 \frac{\partial^4 u}{\partial
%    x^4}(x_{j},t_n)\right) +{\cal O}(\Delta x^4)+ {\cal O}(\Delta
%t^2)\\[2ex]
%\quad =\displaystyle \nu\Delta x^2 \frac{\partial^4 u}{\partial
%    x^4}(x_{j},t_n) +\frac{\nu^2\Delta t \Delta x^2}{2}\frac{\partial^6 u}{\partial
%    x^6}(x_{j},t_n) +{\cal O}(\Delta x^4)+{\cal O}(\Delta
%t^2).
%\end{array}
%\end{equation}
%On constate qu'en ajoutant \`a la formule \eqref{eq:crank}, $1/12$
%de la formule \eqref{eq:reste} on obtient le r\'esultat.

%\item {\it Sch\'ema de Gear}. On choisit cette fois-ci de faire les
%  d\'eveloppements de Taylor autour du point $(x_j,t_{n+1})$. On
%  obtient la formule suivante pour le premier terme (le deuxi\`eme
%  \'etant donn\'e dans \eqref{eq:util})
%$$
%3u(x_j,t_{n+1})-4u(x_j,t_n)+u(x_j,t_{n-1})=2\Delta t \frac{\partial
%  u}{\partial t} (x_j,t_{n+1})+{\cal O}(\Delta t^3),
%$$
%En combinant les deux termes de l'erreur de troncature et en
%utilisation le fait que $u$ est solution de l'\'equation, on obtient
%$$
%\begin{array}{l}
%\displaystyle{\cal E}^n_j=\frac{3u(x_j,t_{n+1})-4u(x_j,t_n)+u(x_j,t_{n-1})}{2\Delta
%  t}-\nu \frac{u(x_{j+1},t_{n+1})-2u(x_j,t_{n+1})+u(x_{j-1},t_{n+1})}{\Delta
%  x^2} \\
%\quad \displaystyle= {\cal O}(\Delta t^2)+{\cal O}(\Delta x^4).
%\end{array}
%$$
\item {\it Sch\'ema de DuFort-Frankel}. En faisant des d\'eveloppement
  de Taylor autour du point $(x_j,t_{n})$ on obtient
$$
\begin{array}{l}
\displaystyle{\cal E}^n_j=\frac{u(x_j,t_{n+1})-u(x_j,t_{n-1})}{2\Delta
  t}-\nu \frac{u(x_{j+1},t_{n})-u(x_j,t_{n+1})-u(x_j,t_{n-1})+u(x_{j-1},t_{n})}{\Delta
  x^2} \\
\quad = \displaystyle \frac{\partial
  u}{\partial t} (x_j,t_{n}) + {\cal O}(\Delta t^2) - \nu \left(2\frac{u(t^n,x_j)}{\Delta x^2}+ \frac{\partial ^2u}{\partial x^2} + {\cal O}(\Delta x^2) \right)
 + \nu \frac{u(x_j,t_{n+1})+u(x_j,t_{n-1})}{\Delta
  x^2} \\
   = \displaystyle \frac{\partial
  u}{\partial t} (x_j,t_{n}) -\nu \frac{\partial ^2u}{\partial x^2} (x_j,t_{n}) 
 + \nu \frac{u(x_j,t_{n+1})+u(x_j,t_{n-1})-2u(x_j,t_n)}{\Delta
  x^2} + {\cal O}(\Delta t^2)+{\cal O}(\Delta x^2)\\
   =\displaystyle \frac{\Delta t^2}{\Delta x^2} \left( \frac{\partial^2 u}{\partial t^2}(x_j,t_n) +{\cal O}(\Delta t^2) \right)+ {\cal O}(\Delta t^2)+{\cal O}(\Delta x^2)
  
%  \nu \displaystyle \frac{u(x_{j+1},t_{n+1})-2u(x_j,t_{n+1})+u(x_{j-1},t_{n+1})}{\Delta
%  x^2} \\
%\quad- \nu \displaystyle \frac{2u(x_j,t_{n+1}) -u(x_j,t_{n+1})-u(x_j,t_{n-1})}{\Delta
%  x^2} = \displaystyle\frac{\partial
%  u}{\partial t} (x_j,t_{n}) + {\cal O}(\Delta t^2) \\
%\quad -\displaystyle \nu\left(\frac{\partial^2  u}{\partial x^2}
%  (x_j,t_{n}) + {\cal O}(\Delta x^2)\right)+\nu \frac{\Delta
%  t^2}{\Delta x^2}\left(\frac{\partial^2  u}{\partial t^2} (x_j,t_{n})
%  + {\cal O}(\Delta t^2)\right)\\
%\quad \displaystyle=\nu \frac{\Delta
%  t^2}{\Delta x^2}\left(\frac{\partial^2  u}{\partial t^2} (x_j,t_{n}) + {\cal O}(\Delta t^2)\right)+{\cal O}(\Delta t^2)+{\cal O}(\Delta x^2).
\end{array}
$$
On voit que le sch\'ema est consistant uniquement si $\frac{\Delta
  t}{\Delta x}$ tend vers $0$.  Dans ce cas on aurait l'ordre $2$ en temps et en espace.

\end{enumerate}

\noindent {\bf Stabilit\'e en norme $L^2$}
\begin{enumerate}
%\item Sch\'ema d'Euler implicite. On injecte un mode de Fourier
%  $u_j^n=A(k)^ne^{2i\pi jk\Delta x}$ dans le sch\'ema, afin de calculer
%  son facteur d'amplification
%$$
%\frac{ A(k)^{n+1}e^{2i\pi jk\Delta x}- A(k)^{n}e^{2i\pi jk\Delta
%    x}}{\Delta t}-\nu\frac{A(k)^{n+1}e^{2i\pi (j+1)k\Delta
%    x}-2 A(k)^{n+1}e^{2i\pi jk\Delta
%    x} +A(k)^{n+1}e^{2i\pi (j-1)k\Delta
%    x}}{\Delta x^2}=0.
%$$
%En simplifiant le facteur $A(k)^ne^{2i\pi jk\Delta x}$ on obtient
%$$
%A(k)-1-\frac{\nu\Delta t}{\Delta x^2}A(k)(e^{2i\pi k\Delta  x} -2 +e^{-2i\pi k\Delta  x})=0.
%$$
%Ensuite,
%$$
%\displaystyle A(k)\left(1 +\frac{\nu\Delta t}{\Delta x^2}(2-2\cos(2\pi\Delta
%  x))\right) =1 \Leftrightarrow A(k) = \frac{1}{1+4
%  \displaystyle\frac{\nu\Delta t}{\Delta x^2}\sin^2(\pi k\Delta x)}
%$$
%ce qui prouve qu'ind\'ependamment de $\Delta t$ et $\Delta x$,
%$|A(k)|\le 1$, donc le sch\'ema est {\it inconditionnellement stable}.
\item Le $\theta$-sch\'ema. On injecte un mode de Fourier
  $u_j^n=A(k)^ne^{2i\pi j k\Delta x}$ dans le sch\'ema, afin de calculer
  son facteur d'amplification
$$
\begin{array}{l}
\displaystyle\frac{ A(k)^{n+1}e^{2i\pi jk\Delta x}- A(k)^{n}e^{2i\pi jk\Delta
    x}}{\Delta t} \\[2ex]
\qquad - (1-\theta)\displaystyle\displaystyle\nu\frac{A(k)^{n+1}e^{2i\pi (j+1)k\Delta
    x}-2 A(k)^{n+1}e^{2i\pi jk\Delta
    x} +A(k)^{n+1}e^{2i\pi (j-1)k\Delta
    x}}{\Delta x^2}\\[2ex]
\qquad- \displaystyle \theta\displaystyle\nu\frac{A(k)^{n}e^{2i\pi (j+1)k\Delta
    x}-2 A(k)^{n}e^{2i\pi jk\Delta
    x} +A(k)^{n}e^{2i\pi (j-1)k\Delta
    x}}{\Delta x^2}=0
\end{array}
$$
En simplifiant le facteur $A(k)^ne^{2i\pi jk\Delta x}$ on obtient
$$
\begin{array}{l}
\displaystyle A(k)-1-(1-\theta)\frac{\nu\Delta t}{\Delta x^2}A(k)(e^{2i\pi
  k \Delta  x} -2
+e^{-2i\pi k\Delta  x})-\theta\frac{\nu\Delta t}{\Delta x^2}(e^{2i\pi k\Delta  x} -2
+e^{-2i\pi k\Delta  x})=0.
\end{array}
$$
ce qui conduit \`a
$$
A(k)\left(1+4(1-\theta)\displaystyle\frac{\nu\Delta t}{\Delta
    x^2}\sin^2(\pi k\Delta x)\right)=1-4\theta
\displaystyle\frac{\nu\Delta t}{\Delta x^2}\sin^2(\pi k\Delta x).
$$
La condition $A(k)\le 1$ sera donc \'equivalente \`a
$$
-1\le \frac{1-4\theta \displaystyle\frac{\nu\Delta t}{\Delta
    x^2}\sin^2(\pi k\Delta x)}{1+4(1-\theta) \displaystyle\frac{\nu\Delta
    t}{\Delta x^2}\sin^2(\pi k\Delta x)}\le 1 \Leftrightarrow 2(2\theta-1) \displaystyle\frac{\nu\Delta
    t}{\Delta x^2}\sin^2(\pi k\Delta x)\le 1.
$$
On en d\'eduit que si $\theta \le 1/2$, le sch\'ema est
{\it inconditionnellement stable} et que si $1/2 < \theta \le 1$ alors il est
stable sous la condition $$2(2\theta-1)\frac{\nu\Delta
    t}{\Delta x^2} \le 1.$$ 
    
    En particulier le schéma de Crank-Nicolson est inconditionnellement stable.
\item Sch\'ema de DuFort-Frankel. On injecte un mode de Fourier
  $u_j^n=A(k)^ne^{2i\pi j k\Delta x}$ dans le sch\'ema, afin de calculer
  son facteur d'amplification et on simplifiera ensuite $A(k)^{n-1}e^{2i\pi jk\Delta x}$
$$
\begin{array}{l}
A(k)^2-1-c\left(2A(k)\cos(k\pi\Delta
  x)-A(k)^2-1\right)=0,\, c=\frac{2\nu\Delta t}{\Delta x^2}\\[2ex]
\qquad \Rightarrow A(k)^2(1+c)-2cA(k)\cos(k\pi\Delta x)+c-1=0.
\end{array}
$$
Il s'agit d'une equation de second degr\'e, possedant $2$ racines
$A_{1,2}(k)$. Si le determinant de cette \'equation est n\'egatif, les
deux racines sont conjugu\'ees complexes, de m\^eme module et 
$$
|A_1(k)|^2= |A_2(k)|^2=|A_1(k)A_2(k)|=\left|\frac{c-1}{c+1}\right|<1.
$$
on en d\'eduit que le sch\'ema est inconditionnellement stable. Si le
determinant est positif, les deux racines sont r\'eelles
$$
A_{1,2}(k)=\frac{c\cos(k\pi\Delta x)\pm \sqrt{c^2\cos^2(k\pi\Delta x)-c^2+1} }{c+1}.
$$
et on pourra montrer facilement par simple calcul que $\max
\{A_1(k),A_2(k)\} = A_1(k) \le  1$ et que $\min
\{A_1(k),A_2(k)\} = A_2(k) \ge -1$.
\end{enumerate}

\end{document}
