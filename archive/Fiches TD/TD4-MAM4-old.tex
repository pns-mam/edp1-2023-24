\documentclass[12pt,a4paper]{article}

%\usepackage[T1]{fontenc} % Pour la bonne cesure du francais
\usepackage{amsmath} % Pour les symboles complementaire comme les matrices !
\usepackage{amssymb}
\usepackage{verbatim}
\usepackage{epsfig}
%\usepackage{/home/cohen/fortran/graphics/GGGraphics/GGGraphics}
%\usepackage{D:/GGGraphics/GGGraphics}

\newtheorem{theorem}{Theorem}
\newtheorem{corollary}[theorem]{Corollary}
\newtheorem{example}{Example}
\newtheorem{rem}{\noindent\textbf{\textit {Remarque\,}}}
\newcommand{\qed}{\hfill$\qedsquare$\goodbreak\bigskip}

\def\e{{\mathchoice{\hbox{\mathbb{R}m e}}{\hbox{\mathbb{R}m e}}%
        {\hbox{\mathbb{R}m \scriptsize e}}{\hbox{\mathbb{R}m \tiny e}}}}
        
\advance\voffset by -35mm \advance\hoffset by -25mm
\setlength{\textwidth}{175mm} \setlength{\textheight}{260mm}
\pagestyle{empty}

\begin{document}

\noindent {\large Universit\'e de Nice-Sophia Antipolis} \hfill EPU\\
\noindent Math\'ematiques Appliqu\'ees et Mod\'elisation (MAM4/SI4) \hfill mercredi 3 octobre 2012 \\

\hrule

\bigskip
\bigskip

\begin{center}{\bf \'Equations aux d\'eriv\'ees partielles --
S\'erie 4}\end{center}

\bigskip

\parskip 12pt
\noindent Le but de cette s\'erie d'exercices est d'illuster la
propri\'et\'e de stabilit\'e en norme $L^\infty$ et en norme $L^2$ de quelques sch\'emas appliqu\'es \`a l'\'equation de la
chaleur.  Pour la stabilit\'e  $L^\infty$ on v\'erifiera le principe
du maximum discret. Pour la stabilit\'e $L^2$ on calculera le facteur
d'amplification de chaque sch\'ema en y injectant un mode de Fourier
et on \'etablira sous quelles conditions ce facteur d'amplification
est inferieur \`a un en module.
 
\noindent Consid\'erons l'\'equation de la chaleur en une dimension d'espace dans la domaine born\'e $(0,1)$:
$$
\begin{cases}
\displaystyle\frac{\partial u}{\partial t}-\nu \frac{\partial^2 u}{\partial x^2}=0,\, \forall (x,t)\in(0,1)\times\mathbb{R}^+_*,\\[2ex]
u(x,0)=u_0(x),\,\forall x\in (0,1).
\end{cases}
$$
On discr\'etise le domaine en utilisant un maillage r\'egulier
$(t_n,x_j)=(n\Delta t,j\Delta x)$,  $\forall n\ge 0,
j\in\{0,1,...,N+1\}$ o\`u $\Delta x=1/(N+1)$ et $\Delta t>0$. Les
conditions aux limites sont des conditions de Dirichlet homog\`enes:
$u(0,t)=u(1,t)=0,\,\forall t\in\mathbb{R}^+_*$. 

\noindent On se propose d'\'etudier les sch\'emas suivants:

\begin{enumerate}
\item {\it Sch\'ema d'Euler explicite}
$$
\frac{u_j^{n+1}-u_j^n}{\Delta t}-\nu \frac{u_{j+1}^{n}-2u_j^n+u_{j-1}^{n}}{\Delta x^2}=0.
$$

\item {\it Sch\'ema d'Euler implicite}
$$
\frac{u_j^{n+1}-u_j^n}{\Delta t}-\nu \frac{u_{j+1}^{n+1}-2u_j^{n+1}+u_{j-1}^{n+1}}{\Delta x^2}=0.
$$

\item {\it Sch\'ema de Crank-Nicolson}
$$
\frac{u_j^{n+1}-u_j^n}{\Delta t}-\nu \frac{u_{j+1}^{n}-2u_j^n+u_{j-1}^{n}}{2\Delta x^2}-\nu \frac{u_{j+1}^{n+1}-2u_j^{n+1}+u_{j-1}^{n+1}}{2\Delta x^2}=0.
$$

\item {\it Le $\theta$ sch\'ema}
$$
\frac{u_j^{n+1}-u_j^n}{\Delta t}-\theta\nu \frac{u_{j+1}^{n}-2u_j^n+u_{j-1}^{n}}{\Delta x^2}-(1-\theta)\nu \frac{u_{j+1}^{n+1}-2u_j^{n+1}+u_{j-1}^{n+1}}{\Delta x^2}=0.
$$

\item {\it Sch\'ema aux six points}
$$
\begin{array}{lcl}
\displaystyle\frac{u_{j+1}^{n+1}-u_{j+1}^n}{12\Delta t}& +& \displaystyle\frac{5(u_{j}^{n+1}-u_{j}^n)}{6\Delta t}+\frac{u_{j-1}^{n+1}-u_{j-1}^n}{12\Delta t}+\\[2ex]
& -&\displaystyle\nu \frac{u_{j+1}^{n}-2u_j^n+u_{j-1}^{n}}{2\Delta x^2}-\nu \frac{u_{j+1}^{n+1}-2u_j^{n+1}+u_{j-1}^{n+1}}{2\Delta x^2}=0.
\end{array}
$$

\item {\it Sch\'ema de DuFort-Frankel}
$$
\frac{u_j^{n+1}-u_j^{n-1}}{2\Delta t}-\nu \frac{u_{j+1}^{n}-u_j^{n+1}-u_j^{n-1}+u_{j-1}^{n}}{\Delta x^2}=0.
$$


{\it Stabilit\'e en norme $L^{\infty}$}\\
1) Montrer que le sch\'ema de Crank-Nicolson est stable en norme $L^{\infty}$ si $\nu\Delta t\le \Delta x^2$.\\
2) Montrer que que le sch\'ema de DuFort-Frankel et Euler explicite sont stables en norme $L^{\infty}$ si $2\nu\Delta t\le \Delta x^2$. \\
3) Que peut-on dire de la stabilit\'e $L^{\infty}$ du sch\'ema d'Euler
implicite?\\

{\it Stabilit\'e en norme $L^2$}\\
1) Montrer que le sch\'ema explicite est stable en norme $L^2$ sous la condition $2\nu\Delta t\le \Delta x^2$ et que le sch\'ema implicite est inconditionnellement stable.\\
2) Montrer que le $\theta$- sch\'ema est stable en norme $L^2$ inconditionnellement si $1/2\le \theta \le 1$ et sous la conditions CFL $2(1-2\theta)\nu\Delta t\le (\Delta x)^2$ si $0\le\theta\le 1$.\\
3) Montrer que le sch\'ema de DuFort-Frankel est inconditionnellement
stable en norme $L^2$. Montrer que si on fait tendre $\Delta t$ et
$\Delta x$ simultanement vers $0$ de mani\`ere que le rapport $\Delta
t/\Delta x$ tende aussi vers $0$, alors le sch\'ema de DuFort-Frankel
est convergent (on dit qu'il est "conditionnellement'' convergent).  

\end{enumerate}

\bigskip
\hrule
\noindent{\bf Evaluation du cours \'Equations aux D\'eriv\'ees Partielles :}
\begin{itemize}
\item[$\bullet$] Un contr\^ole \'ecrit le mercredi $24$ octobre (pendant la s\'eance de TD). 
\item[$\bullet$] Une note sur les exercices des TDs: pr\'esence au
  tableau pendant les s\'eances.
\item[$\bullet$] Un examen \'ecrit durant la session d'examens. 
\end{itemize}
La note finale est : $\frac{1}{3}$(note contr\^ole cont.) $+$
$\frac{1}{3}$(note exercices TD) $+$
$\frac{1}{3}$(note examen).

\end{document}
