\documentclass[12pt,a4paper]{article}

%\usepackage[T1]{fontenc} % Pour la bonne cesure du francais
\usepackage{amsmath} % Pour les symboles complementaire comme les matrices !
\usepackage{amssymb}
\usepackage{verbatim}
\usepackage{epsfig}
%\usepackage{/home/cohen/fortran/graphics/GGGraphics/GGGraphics}
%\usepackage{D:/GGGraphics/GGGraphics}

\newtheorem{theorem}{Theorem}
\newtheorem{corollary}[theorem]{Corollary}
\newtheorem{example}{Example}
\newtheorem{rem}{\noindent\textbf{\textit {Remarque\,}}}
\newcommand{\qed}{\hfill$\qedsquare$\goodbreak\bigskip}

\def\e{{\mathchoice{\hbox{\mathbb{R}m e}}{\hbox{\mathbb{R}m e}}%
        {\hbox{\mathbb{R}m \scriptsize e}}{\hbox{\mathbb{R}m \tiny e}}}}
        
\advance\voffset by -35mm \advance\hoffset by -25mm
\setlength{\textwidth}{175mm} \setlength{\textheight}{260mm}
\pagestyle{empty}

\begin{document}

\noindent {\large Universit\'e de Nice-Sophia Antipolis} \hfill EPU\\
\noindent Math\'ematiques Appliqu\'ees et Mod\'elisation (MAM4/SI4) \hfill mercredi 10 octobre 2012 \\

\hrule

\bigskip
\bigskip

\begin{center}{\bf \'Equations aux d\'eriv\'ees partielles --
S\'erie 4 \\
SOLUTIONS}\end{center}

\bigskip

\parskip 12pt
{\it Stabilit\'e en norme $L^{\infty}$}.
\begin{enumerate}
\item Sch\'ema de Crank-Nicolson. On notera comme avant par
  $M=\max_{j\in\mathbb{Z}}\{u_j^{n+1}\}$ et
  $m=\min_{j\in\mathbb{Z}}\{u_j^{n+1}\}$ le maximum et le minimum des
  valeurs fournies par le sch\'ema \`a l'instant $n+1$. On montrera
  d'abord que le maximum est born\'e par les valeurs obtenues \`a
  l'instant pr\'ecedent. Notos par
  $u_k^{n+1}$, l'\'el\'ement pour lequel ce maximum est atteint. On
  \'ecrit ensuite le  sch\'ema pour $j=k$ et on utilise le fait que
  le maximum est $u_k^{n+1}$. Ceci nous donne
$$
\frac{M-u_k^n}{\Delta t}-\nu\frac{u_{k-1}^n-u_k^n+u_{k+1}^n}{2\Delta
  x^2} \le 0 \Rightarrow M\le \left(1-\frac{\nu\Delta t}{\Delta
    x^2}\right)u_k^n +\frac{\nu\Delta t}{2\Delta
    x^2} (u_{k-1}^n+u_{k+1}^n).
$$
On voit bien que si $\nu\Delta t\le \Delta x^2$, le terme de droite
est une combinaison convexe des $u^n_{k-1}$, $u^n_k$ et $u^n_{k+1}$,
d'o\`u le fait que $M\le \max_{j\in \mathbb{Z}}\{u_j^n\}$. En ce qui
concerne l'in\'egalit\'e sur le minimum, ceci se d\'eduit avec les
m\^emes arguments, en rempla\c{c}ant $u^n_j$ par $-u_j^n$ et $M$ par $-m$. 
\item Sch\'ema de DuFort-Frankel. On va r\'e-\'ecrire ce sch\'ema de
  la fa\c{c}on suivante.
$$
\left(\frac{1}{2\Delta t}+\frac{\nu}{\Delta
    x^2}\right)u_j^{n+1}=\left(\frac{1}{2\Delta t}-\frac{\nu}{\Delta
    x^2}\right) u_j^{n-1}+\frac{\nu}{\Delta x^2}(u^n_{j-1}+u^n_{j+1}).
$$
On remarque que si $2\nu\Delta t\le \Delta x^2$, alors $u_j^{n+1}$ est
bien une combinaison convexe des \'el\'ements du second membre, ce qui
conduit \`a la stabilit\'e $L^\infty$ su sch\'ema.
\end{enumerate}

{\it Stabilit\'e en norme $L^2$}
\begin{enumerate}
\item Sch\'ema d'Euler implicite. On injecte un mode de Fourier
  $u_j^n=G(k)^ne^{2i\pi jk\Delta x}$ dans le sch\'ema, afin de calculer
  son facteur d'amplification
$$
\frac{ G(k)^{n+1}e^{2i\pi jk\Delta x}- G(k)^{n}e^{2i\pi jk\Delta
    x}}{\Delta t}-\nu\frac{G(k)^{n+1}e^{2i\pi (j+1)k\Delta
    x}-2 G(k)^{n+1}e^{2i\pi jk\Delta
    x} +G(k)^{n+1}e^{2i\pi (j-1)k\Delta
    x}}{\Delta x^2}=0.
$$
En simplifiant le facteur $G(k)^ne^{2i\pi jk\Delta x}$ on obtient
$$
G(k)-1-\frac{\nu\Delta t}{\Delta x^2}G(k)(e^{2i\pi k\Delta  x} -2 +e^{-2i\pi k\Delta  x})=0.
$$
Ensuite,
$$
\displaystyle G(k)\left(1 +\frac{\nu\Delta t}{\Delta x^2}(2-2\cos(2\pi\Delta
  x))\right) =1 \Leftrightarrow G(k) = \frac{1}{1+4
  \displaystyle\frac{\nu\Delta t}{\Delta x^2}\sin^2(\pi k\Delta x)}
$$
ce qui prouve qu'ind\'ependamment de $\Delta t$ et $\Delta x$,
$|G(k)|\le 1$, donc le sch\'ema est {\it inconditionnellement stable}.
\item Le $\theta$-sch\'ema. On injecte un mode de Fourier
  $u_j^n=G(k)^ne^{2i\pi j k\Delta x}$ dans le sch\'ema, afin de calculer
  son facteur d'amplification
$$
\begin{array}{l}
\displaystyle\frac{ G(k)^{n+1}e^{2i\pi jk\Delta x}- G(k)^{n}e^{2i\pi jk\Delta
    x}}{\Delta t} \\[2ex]
\qquad - \theta\displaystyle\displaystyle\nu\frac{G(k)^{n+1}e^{2i\pi (j+1)k\Delta
    x}-2 G(k)^{n+1}e^{2i\pi jk\Delta
    x} +G(k)^{n+1}e^{2i\pi (j-1)k\Delta
    x}}{\Delta x^2}\\[2ex]
\qquad- \displaystyle (1-\theta)\displaystyle\nu\frac{G(k)^{n}e^{2i\pi (j+1)k\Delta
    x}-2 G(k)^{n}e^{2i\pi jk\Delta
    x} +G(k)^{n}e^{2i\pi (j-1)k\Delta
    x}}{\Delta x^2}=0
\end{array}
$$
En simplifiant le facteur $G(k)^ne^{2i\pi jk\Delta x}$ on obtient
$$
\begin{array}{l}
\displaystyle G(k)-1-\theta\frac{\nu\Delta t}{\Delta x^2}G(k)(e^{2i\pi
  k \Delta  x} -2
+e^{-2i\pi k\Delta  x})-(1-\theta)\frac{\nu\Delta t}{\Delta x^2}(e^{2i\pi k\Delta  x} -2
+e^{-2i\pi k\Delta  x})=0.
\end{array}
$$
ce qui conduit \`a
$$
G(k)\left(1+4\theta \displaystyle\frac{\nu\Delta t}{\Delta
    x^2}\sin^2(\pi k\Delta x)\right)=1-4(1-\theta)
\displaystyle\frac{\nu\Delta t}{\Delta x^2}\sin^2(\pi k\Delta x).
$$
La condition $G(k)\le 1$ sera donc \'equivalente \`a
$$
-1\le \frac{1-4(1-\theta) \displaystyle\frac{\nu\Delta t}{\Delta
    x^2}\sin^2(\pi k\Delta x)}{1+4\theta \displaystyle\frac{\nu\Delta
    t}{\Delta x^2}\sin^2(\pi k\Delta x)}\le 1 \Leftrightarrow (1-2\theta) \displaystyle\frac{\nu\Delta
    t}{\Delta x^2}\sin^2(\pi k\Delta x)\le 1.
$$
On en d\'eduit que si $\theta \ge 1/2$, le sch\'ema est
{\it inconditionnellement stable} et que si $0\le \theta < 1/2$ alors il est
stable sous la condition $(1-2\theta)\frac{\nu\Delta
    t}{\Delta x^2} \le 1$.
\item Sch\'ema de DuFort-Frankel. On injecte un mode de Fourier
  $u_j^n=G(k)^ne^{2i\pi j k\Delta x}$ dans le sch\'ema, afin de calculer
  son facteur d'amplification et on simplifiera ensuite $G(k)^{n-1}e^{2i\pi jk\Delta x}$
$$
\begin{array}{l}
G(k)^2-1-c\left(2G(k)\cos(k\pi\Delta
  x)-G(k)^2-1\right)=0,\, c=\frac{2\nu\Delta t}{\Delta x^2}\\[2ex]
\qquad \Rightarrow G(k)^2(1+c)-2cG(k)\cos(k\pi\Delta x)+c-1=0.
\end{array}
$$
Il s'agit d'une equation de second degr\'e, possedant $2$ racines
$G_{1,2}(k)$. Si le determinant de cette \'equation est n\'egatif, les
deux racines sont conjugu\'ees complexes, de m\^eme module et 
$$
|G_1(k)|^2= |G_2(k)|^2=|G_1(k)G_2(k)|=\left|\frac{c-1}{c+1}\right|<1.
$$
on en d\'eduit que le sch\'ema est inconditionnellement stable. Si le
determinant est positif, les deux racines sont r\'eelles
$$
G_{1,2}(k)=\frac{c\cos(k\pi\Delta x)\pm \sqrt{c^2\cos^2(k\pi\Delta x)-c^2+1} }{c+1}.
$$
et on pourra montrer facilement par simple calcul que $\max
\{G_1(k),G_2(k)\} = G_1(k) \le  1$ et que $\min
\{G_1(k),G_2(k)\} = G_2(k) \ge -1$.
\end{enumerate}
\end{document}
