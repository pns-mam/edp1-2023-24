\documentclass[12pt,a4paper]{article}

%\usepackage[T1]{fontenc} % Pour la bonne cesure du francais
\usepackage{amsmath} % Pour les symboles complementaire comme les matrices !
\usepackage{amssymb}
\usepackage{verbatim}
\usepackage{epsfig}
%\usepackage{/home/cohen/fortran/graphics/GGGraphics/GGGraphics}
%\usepackage{D:/GGGraphics/GGGraphics}

\newtheorem{theorem}{Theorem}
\newtheorem{corollary}[theorem]{Corollary}
\newtheorem{example}{Example}
\newtheorem{rem}{\noindent\textbf{\textit {Remarque\,}}}
\newcommand{\qed}{\hfill$\qedsquare$\goodbreak\bigskip}
\newcommand{\dxx}{\frac{\partial}{\partial x}}
\newcommand{\dxy}{\frac{\partial }{\partial y}}
\newcommand{\dxz}{\frac{\partial}{\partial z}}
\newcommand{\dxi}{\frac{\partial}{\partial x_i}}
\newcommand{\Dxy}{\frac{\partial^2}{\partial x \partial y}}
\newcommand{\Dxz}{\frac{\partial^2}{\partial x \partial z}}
\newcommand{\Dyz}{\frac{\partial^2}{\partial y \partial z}}
\newcommand{\uu}{\mathbf{u}}
\newcommand{\dr}{\frac{\partial}{\partial r}}
\newcommand{\drr}{\frac{\partial^2}{\partial r^2}}
\newcommand{\de}{\frac{\partial}{\partial \theta}}
\newcommand{\dt}{\frac{\partial}{\partial t}}
\newcommand{\doo}{\frac{\partial^2}{\partial \theta^2}}
\newcommand{\xx}{\mathbf{x}}
\newcommand{\kk}{\mathbf{k}}
\newcommand{\XX}{\mathbf{X}}
\newcommand{\xp}{\mathbf{x'}}
\newcommand{\EE}{\mathbf{E}}
\newcommand{\HH}{\mathbf{H}}
\newcommand{\nn}{\mathbf{n}}
\newcommand{\cE}{\widehat{\mathbf{E}}}
\newcommand{\cH}{\widehat{\mathbf{H}}}
\newcommand{\drx}{\frac{\partial r}{\partial x}}
\newcommand{\dry}{\frac{\partial r}{\partial y}}
\newcommand{\dox}{\frac{\partial \theta}{\partial x}}
\newcommand{\doy}{\frac{\partial \theta}{\partial y}}
\newcommand{\dur}{\frac{\partial u}{\partial r}}
\newcommand{\duo}{\frac{\partial u}{\partial \theta}}
\newcommand{\dx}{\frac{\partial }{\partial x}}
\newcommand{\dy}{\frac{\partial }{\partial y}}
\newcommand{\duro}{\frac{\partial^2 u}{\partial r\partial \theta}}
\newcommand{\durr}{\frac{\partial^2 u}{\partial r^2}}
\newcommand{\duoo}{\frac{\partial^2 u}{\partial \theta^2}}
\newcommand{\sincos}{\sin\theta\cos\theta}

\def\e{{\mathchoice{\hbox{\mathbb{R}m e}}{\hbox{\mathbb{R}m e}}%
        {\hbox{\mathbb{R}m \scriptsize e}}{\hbox{\mathbb{R}m \tiny e}}}}
        
\advance\voffset by -35mm \advance\hoffset by -25mm
\setlength{\textwidth}{175mm} \setlength{\textheight}{260mm}
\pagestyle{empty}

\begin{document}

\noindent {\large Universit\'e C\^ote d'Azur} \hfill Polytech' Nice Sophia (PNS)\\
\noindent Math\'ematiques Appliqu\'ees et Mod\'elisation (MAM4) \hfill vendredi 20 Novembre 2020 \\

\hrule

\bigskip
\bigskip

\begin{center}{\bf \'Equations aux d\'eriv\'ees partielles --
SOLUTIONS}\end{center}

\bigskip
On va faire des calculs et si c'est n\'ecessaire on va donner des explications additionnelles 
\begin{enumerate}
\item 
$$
\begin{array}{ccl}
  \nabla \cdot (p \mathbf{v})&=& \dxx (p v_x)+\dxy (p v_y)+\dxz (p v_z)\\
  &=& v_x \dxx p+v_y \dxy p +v_z \dxz p +p (\dxx v_x+\dxy v_y+\dxz v_z)\\
  &=& \nabla p\cdot \mathbf{v}+p\nabla\cdot \mathbf{v}.
  \end{array}
$$

 $$\begin{array}{ccl}
  \nabla\times (p\mathbf{v})&=&(\dxy(pv_z)-\dxz (pv_y),\dxz (pv_x)-\dxx (pv_z),\dxx (pv_y)-\dxy (pv_x))^T\\
  &=& \left(\begin{array}{c}
  			\dxy p \,v_z -\dxz p \, v_y+p(\dxy v_z-\dxz v_y)\\
  			\dxz p \,v_x -\dxx p \, v_x+p(\dxz v_x-\dxx v_z)\\
  			\dxx p \,v_y -\dxy p \, v_x+p(\dxx v_y-\dxy v_x)		
  	      \end{array}\right)\\
  &=& \nabla p\times \mathbf{v}+p(\nabla \times \mathbf{v}).
  \end{array}
$$

$$
\begin{array}{ccl}\nabla\times (\nabla p)&=&\nabla\times (\dxx p,\dxy p, \dxz p)\\
	&=& (\Dyz p-\Dyz p,\Dxz p-\Dxz p,\Dxy p -\Dxy p)=0
	\end{array}
$$
 $$
\begin{array}{ccl}
	\nabla \cdot (\nabla \times \mathbf{v})&=& \nabla \cdot (\dxy v_z-\dxz v_y\, ,\dxz v_x -\dxx v_z\, , \dxx v_y -\dxy v_x)\\
	&=& \Dxy v_z -\Dxz v_y +\Dyz v_x -\Dxy v_z+\Dxz v_y -\Dyz v_x=0
	\end{array}
$$
\item Dans la formule de Green on remplace d'abord $u$  par $uv$ ce qui conduit
  \`a 
$$
\int_\Omega \frac{\partial u}{\partial x_i}  \, v \, d\mathbf{x}= \, - \, \int_\Omega 
u \frac{\partial v}{\partial x_i}\,d\mathbf{x}+ \, \int_{\partial \Omega}
u v n_i \, d\sigma.
$$
On remplacera ensuite $u$ par $\frac{\partial u}{\partial x_i},\,
i=1,..,N$ et puis on sommera les relations. 
\item On va utiliser la formule $\nabla\cdot (p\mathbf{v})=p\nabla
  \cdot \mathbf{v}+\mathbf{v}\cdot \nabla p$ et la formule de Green.
$$\begin{array}{ccl}
	\int_{\Omega}(\nabla\cdot \mathbf{w})\phi d\mathbf{x}&=&\int_{\Omega}\nabla\cdot(\phi \mathbf{w})d\mathbf{x}-\int_{\Omega}\mathbf{w}\cdot \nabla \phi d\mathbf{x}\\
	&=& -\int_{\Omega} \mathbf{w}\cdot \nabla \phi d\mathbf{x}+\int_{\Omega}\sum_{i=1}^3 \dxi (\phi \mathbf{w})d\mathbf{x}\\
&=& -\int_{ \Omega} \mathbf{w}\cdot \nabla \phi d \mathbf{x} +\sum_{i=1}^3 \int_{\partial \Omega} (\phi \mathbf{w} )\cdot n_i(x) d \sigma\\
	&=& -\int_{\Omega} \mathbf{w} \cdot \nabla \phi d\mathbf{x}+\int_{\partial\Omega}(\mathbf{w}\cdot \mathbf{n})\phi d\sigma.
	\end{array}
$$
En prenant $\phi=1$, on a $\nabla \phi =0 $ et donc
	$$\int_{\Omega}\nabla\cdot
        \mathbf{w}d\mathbf{x}=\int_{\partial\Omega}\mathbf{w}\cdot
        \mathbf{n} d\sigma.$$

\item 
On utilise la formule $\nabla\times (p\mathbf{v}) = p \nabla\times \mathbf{v}+ \nabla p \times \mathbf{v}$ et la formule de Green.
	$$
{\begin{array}{ccl}
	 \int_{\Omega}(\nabla\times \mathbf{w})\phi d\mathbf{x}&=& \int_{\Omega}\nabla\times (\phi \mathbf{w})d\mathbf{x}-\int_{\Omega}\nabla\phi\times \mathbf{w} d\mathbf{x}\vspace{0.3cm}\\
	&=& \int_{\Omega} \mathbf{w} \times \nabla\phi d\mathbf{x}+\int_{\Omega}(\dxy (\phi w_3)-\dxz (\phi w_2), \dxz (\phi w_1)-\dxx (\phi w_3), \dxx (\phi w_2)-\dxy (\phi w_1))^T d \mathbf{x}\vspace{0.3cm}\\
	&=& \int_{\Omega} \mathbf{w} \times \nabla\phi d\mathbf{x}+\int_{\partial \Omega} \left(\phi(w_3 n_2-w_2n_3),\phi(w_1n_3-w_3n_1),\phi(w_2n_1-w_1n_2)\right)^T \vspace{0.3cm}\\
	&=&\int_{\Omega} \mathbf{w} \times \nabla\phi d\mathbf{x}+\int_{\partial\Omega}(\mathbf{n}\times \mathbf{w})\phi d\sigma \vspace{0.3cm}\\
	&=&\int_{\Omega} \mathbf{w} \times \nabla\phi d\mathbf{x}-\int_{\partial\Omega}(\mathbf{w}\times \mathbf{n})\phi d\sigma
	\end{array}}
$$
On prend encore une fois $\phi=1$ et on obtient:
	$$\int_{\Omega}(\nabla\times \mathbf{w})\phi d\mathbf{x}=\int_{\partial\Omega}\mathbf{w}\times \mathbf{n}d\sigma.$$
\item On v\'erifiera que $u$ satisafait bien l'\'equation ainsi que
  les conditions aux limites. Il est \'evident que $u(0)=u(1)=0$. On
  \'evalue d'abord la d\'eriv\'ee premi\`ere et ensuite la d\'eriv\'ee seconde
$$
\begin{array}{l}
u'(x)=\int_0^1f(s)(1-s)ds -\int_0^x f(s)ds - xf(x)+
xf(x)=\int_0^1f(s)(1-s)ds -\int_0^x f(s)ds\\
u''(x) = (u'(x))'=-f(x)
\end{array}
$$
donc $u$ est bien solution du probl\`eme aux limites.
\item On va calculer d'abord quelques d\'eriv\'ees partielles qui vont \^etre utiles. 
$$\begin{array}{l}
\displaystyle\drx=\frac{x}{\sqrt{x^2+y^2}}=\frac{r\cos\theta}{r}=\cos\theta,\, \dry=\frac{y}{\sqrt{x^2+y^2}}=\sin\theta\\
\displaystyle\dox=\frac{-y}{x^2+y^2}=-\frac{\sin\theta}{r},\, \doy=\frac{x}{x^2+y^2}=\frac{\cos\theta}{r}
\end{array}$$
Maintenant on calcule le Laplacien
$$
\begin{array}{l}
	\Delta u=\displaystyle\frac{\partial^2 u}{\partial x^2}+\frac{\partial^2 u}{\partial y^2}\\
	\quad=\displaystyle\dx(\dur\drx+\duo\dox)+\dy(\dur\dry+\duo\doy)\\
	\quad=\displaystyle\dx(\dur\cos\theta-\duo \frac{\sin\theta}{r})+\dy(\dur \sin\theta+\duo \frac{\cos\theta}{r})\\
	\quad=\displaystyle\dr(\dur \cos\theta)\drx+\doo(\dur\cos\theta)\dox-\dr(\duo\frac{\sin\theta}{r})\drx-\doo(\duo\frac{\sin \theta}{r})\dox\\
	\quad+\displaystyle\dr(\dur\sin\theta)\dry+\doo(\dur\sin\theta)\doy+\dr(\duo\frac{\cos\theta}{r})\dry+\doo(\duo\frac{\cos\theta}{r})\doy\\
	\quad
        =\displaystyle\durr\cos^2\theta-\frac{\sincos}{r}\duro+\frac{\sin^2
          \theta}{r}\dur\\
\quad \displaystyle-\frac{\sincos}{r}\duro+\frac{\sincos}{r^2}\duo+\frac{\sin^2\theta}{r^2}\duoo+\frac{\sincos}{r^2}\duo\\
	 \quad+\displaystyle\durr
         \sin^2\theta+\frac{\sincos}{r}\duro+\frac{\cos^2\theta}{r}\dur\\
\quad \displaystyle+\frac{\sincos }{r}\duro -\frac{\sincos}{r^2}\duo+\frac{\cos^2\theta}{r^2}\duoo -\frac{\sincos}{r^2}\duo\\
	\quad=\displaystyle\durr+\frac{1}{r}\dur+\frac{1}{r^2}\duoo
	\end{array}
$$
\end{enumerate}


\end{document}
