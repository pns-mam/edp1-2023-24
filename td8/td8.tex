\documentclass[12pt,a4paper]{article}

%\usepackage[T1]{fontenc} % Pour la bonne cesure du francais
\usepackage{amsmath} % Pour les symboles complementaire comme les matrices !
\usepackage{amssymb}
\usepackage{verbatim}
\usepackage{epsfig}
%\usepackage{/home/cohen/fortran/graphics/GGGraphics/GGGraphics}
%\usepackage{D:/GGGraphics/GGGraphics}

\newtheorem{theorem}{Theorem}
\newtheorem{corollary}[theorem]{Corollary}
\newtheorem{example}{Example}
\newtheorem{rem}{\noindent\textbf{\textit {Remarque\,}}}
\newcommand{\qed}{\hfill$\qedsquare$\goodbreak\bigskip}

\def\e{{\mathchoice{\hbox{\mathbb{R}m e}}{\hbox{\mathbb{R}m e}}%
        {\hbox{\mathbb{R}m \scriptsize e}}{\hbox{\mathbb{R}m \tiny e}}}}
        
\advance\voffset by -35mm \advance\hoffset by -25mm
\setlength{\textwidth}{175mm} \setlength{\textheight}{260mm}
\pagestyle{empty}

\begin{document}

\noindent {\large Universit\'e C\^ote d'Azur} \hfill Polytech Nice Sophia (PNS)\\
\noindent Math\'ematiques Appliqu\'ees et Mod\'elisation (MAM4) \hfill 2023-24 \\

\hrule

\bigskip
\bigskip

\begin{center}{\bf \'Equations aux d\'eriv\'ees partielles --
TD 8}\end{center}

\bigskip
\begin{enumerate}
\item On consid\`ere le probl\`eme aux limites de Neumann suivant:
\begin{equation}\label{NeuGen}
\begin{cases}
-\Delta u = f,\,\text{dans }\Omega\\
\frac{\partial u}{\partial n} = g,\,\text{sur }\partial\Omega.
\end{cases}
\end{equation}
o\`u $f$ et $g$ sont continues sur $\bar\Omega$ et $\Omega$ est un ouvert born\'e r\'egulier et connexe.\\
{\bf a}. Trouver une formulation variationnelle (V) de ce probl\`eme sur l'espace $X=H^1(\Omega)$ et montrer qu'une fonction $u\in{\cal C}^2(\bar\Omega)$ est solution de (\ref{NeuGen}) ssi est solution de (V).\\
{\bf b}. Montrer qu'une condition n\'ecessaire d'existence de solution
de (\ref{NeuGen}) portant sur $f$ et $g$ (condition dite de
compatibilit\'e) est la suivante:
$$
\int_{\Omega}f(x)dx+ \int_{\partial\Omega} g(x)d\sigma= 0.
$$
{\bf c}. Montrer que $u$ est solution de (V) ssi elle minimise sur $X$ une fonctionnelle $E(v)$ que l'on pr\'ecisera.\\
{\bf d}. Montrer que la solution n'est pas unique à une constante pr\`es. 
%de la solution de (\ref{NeuGen}) dans ${\cal C}^2(\bar\Omega)$ si elle existe.

\item On consid\`ere le probl\`eme des plaques suivant:
\begin{equation}\label{Plaque}
\begin{cases}
\Delta(\Delta u) = f,\,\text{dans }\Omega\\
u = \frac{\partial u}{\partial n} = 0,\,\text{sur }\partial\Omega.
\end{cases}
\end{equation}
o\`u $f$ est continue sur $\bar\Omega$.\\
{\bf a}. Trouver une formulation variationnelle (V) de ce probl\`eme sur l'espace $X=\{v\in H^2(\Omega),\, v=\frac{\partial v}{\partial n}=0,\, \partial\Omega\}$. Montrer qu'une fonction $u\in{\cal C}^4(\bar\Omega)$ est solution de (\ref{Plaque}) ssi elle est solution de (V).\\
{\bf b}. Montrer que $u$ est solution de (V) ssi elle minimise sur $X$ une fonctionnelle $E(v)$ que l'on pr\'ecisera.\\
%{\bf c}. En d\'eduire l'unicit\'e de la solution de (\ref{Plaque}) dans ${\cal C}^4(\bar\Omega)$ si elle existe.

\item On suppose que $\Omega$ est un ouvert borné régulier de classe ${\cal C}^1$. A l'aide de l'approche variationnelle démontrer l'existence et unicité de la solution du problème aux limites

\begin{equation}\label{Robin}
\begin{cases}
-\Delta u = f,\,\text{dans }\Omega\\
\frac{\partial u}{\partial n} + u = g,\,\text{sur }\partial\Omega.
\end{cases}
\end{equation}
où $f\in L^2(\Omega)$. On admettra l'inégalité suivante (qui généralise l'inégalité de Poincaré):
$$
\|v\|_{L^2(\Omega)} \le C (\|v\|_{L^2(\partial\Omega)} +\| \nabla v\|_{L^2(\Omega)} ),\, \forall v \in H^1(\Omega).
$$


\item Soit $\Omega$ un ouvert r\'egulier. On consid\`ere l'\'equation 
\begin{equation}\label{ConvDiff}
\begin{cases}
-\Delta u +V\cdot\nabla u = f,\,\text{dans }\Omega\\
u = 0,\,\text{sur }\partial\Omega.
\end{cases}
\end{equation}
o\`u $V$ est un champ de divergence nulle. \\
{\bf a} Reprendre la premi\`ere question de l'exercice $1$. \\
{\bf b} Peut-on associer à ce probl\`eme un probl\`eme de
minimisation comme nous l'avons fait à l'exercice $1$? \\
{\bf c} Montrer que (\ref{ConvDiff}) admet au plus une solution. \\
{\bf d} Aurait-on obtenu un r\'esultat identique si la divergence de
$V$ prenait des valeurs strictement positives?


%\item Soit $\Omega$ un ouvert de $\mathbb{R}^N$ et $(\Omega_1,\Omega_2)$ une
%  partition de $\Omega$ et $A(x)=k_iId,\,\forall
%  x\in\Omega_i,\,i=1,2$. On note
%  $\Gamma=\partial\Omega_1\cap\partial\Omega_2$ l'interface
%  (suppos\'ee r\'eguli\`ere et incluse dans $\Omega$) entre $\Omega_1$
%  et $\Omega_2$. Montrer que si $u_i\in {\cal C}(\bar \Omega_i),\,i=1,2$ sont solutions de
%$$
%\begin{cases}
%-k_i\Delta u_i=f,\,\text{dans }\Omega_i,\,i=1,2,\\
%u_1=u_2,\,\text{sur }\Gamma,\\
%k_1\nabla u_1\cdot n = k_2\nabla u_2 \cdot n,\,\text{sur }\Gamma,\\
%u_i=0,\,i=1,2,\,\text{sur }\partial\Omega.
%\end{cases}
%$$
%alors la fonction $u$ d\'efinie comme $u_i$ dans $\Omega_i,\,i=1,2$
%est l'unique solution dans $X=\{u\in {\cal C}(\bar \Omega),\,
%u|_{\Omega_i}\in {\cal C}^1(\bar \Omega_i),\, i=1,2, \, u=0 \mbox{
%  sur } \partial\Omega \}$ de 
%$$
%\begin{cases}
%-\text{div}(A\nabla u)=f,\,\text{dans }\Omega,\\
%u=0,\,\text{sur }\partial\Omega.
%\end{cases}
%$$
\end{enumerate}

%\bigskip
%\hrule
%\noindent{\bf Evaluation du cours \'Equations aux D\'eriv\'ees Partielles :}
%\begin{itemize}
%\item[$\bullet$] Un contr\^ole \'ecrit le vendredi $23$ Octobre (pendant la s\'eance de cours). 
%\item[$\bullet$] Une note de devoir maison/projet (présentation orale le 4 décembre).
%\item[$\bullet$] Un examen \'ecrit pendant la session d'examen. 
%\end{itemize}
%La note finale est : $30\%$(note contr\^ole) $+$
%$30\%$(note devoir/projet) $+$
%$40\%$(note examen).
\end{document}
